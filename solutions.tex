\documentclass{article}

\usepackage{fancyhdr}
\usepackage{extramarks}
\usepackage{amsmath}
\usepackage{amsthm}
\usepackage{amssymb}
\usepackage{amsfonts}

\newcommand{\w}[0]{\omega}
\newcommand{\z}[0]{\zeta}

\begin{document}

\section*{Chapter 2}

\begin{enumerate}

\item[1]
\begin{enumerate}
    \item[(a)] Show every number field of degree 2 over $\mathbb{Q}$ is one of the quadratic fields.

    Let $K$ be a number field of degree 2, and $f(x) = x^2 + px + q$ be its minimum polynomial over $\mathbb{Q}$.  Since $p, q \in \mathbb{Q}$ we can multiply through to clear the denominators and give us a polynomial $g(x) = ax^2 + bx + c$ over $\mathbb{Z}$ with the same roots as $f(x)$.  Therefore $K = \mathbb{Q}[\sqrt{b^2 - 4ac}]$ is a quadratic field for $m = b^2 - 4ac$.

    \item[(b)] Suppose $K = \mathbb{Q}[\sqrt{m}]$ contains $\sqrt{n}$ for $n$ a squarefree integer.  Since $K$ has the basis $\{1, \sqrt{m}\}$, so $\sqrt{n} = p + q\sqrt{m}$ for $p, q \in \mathbb{Q}$. Therefore $n = p^2 + 2pq\sqrt{m} + q^2m$, so either $p = 0$ or $q = 0$.

    If $p = 0$, then $\sqrt{n} = q\sqrt{m}$ and so $\sqrt{n} / \sqrt{m} = q$.  This can only happen if $q = 1$, meaning $m = n$.

    If $q = 0$, then $\sqrt{n} = p$, which can only happen if $p$ is also an integer, contradicting $n$ squarefree.

    Therefore the quadratic fields are each distinct.
\end{enumerate}

\item[2]
    Let $I$ be the ideal generated by $2$ and $1 + \sqrt{-3}$ in the ring $\mathbb{Z}[\sqrt{-3}]$.

    We have $I \neq (2)$ because $1 + \sqrt{-3}$ ($\in I$) does not have the form $2a + b\sqrt{-3}$ for $a, b \in \mathbb{Z}$.  The ideal $I^2$ is generated by $(4, 2 + 2\sqrt{-3}, -2 + 2\sqrt{-3})$.  The number $-2 + 2\sqrt{-3} = 2 + 2\sqrt{-3} - 4$ and so is redundant as a generator; therefore $I^2 = (4, 2 + 2\sqrt{-3}) = 2I$.

    Since $I^2 = 2I$, prime factorization of ideals in $\mathbb{Z}[\sqrt{-3}]$ must not hold; if we did then $I$ would be invertible, meaning it could be cancelled from the right-hand-side of each equality, giving us $I = (2)$ which is not true (from above).

    Suppose $P$ is a prime ideal of $\mathbb{Z}[\sqrt{-3}]$ containing $2$.  Then $4 \in P$ also.  Since $(1 + \sqrt{-3})(1 - \sqrt{-3}) = 4$ and $P$ is a prime ideal, one of $1 + \sqrt{-3}$ and $1 - \sqrt{-3}$ are also in $P$.  However, if $1 - \sqrt{-3} \in P$ then $1 + \sqrt{-3} \in P$ since $-1 \cdot (1 - \sqrt{-3}) + 2 = 1 + \sqrt{-3}$.  Therefore any prime ideal containing $(2)$ also contains $I$ and $I$ is the unique prime ideal that contains $(2)$.  Since $I$ cannot be expressed as a product of prime ideals, neither can $(2)$.

    (We should expect this; $\mathbb{Z}[\sqrt{-3}]$ is an order of conductor $2$ in $\mathbb{Z}[\frac{1 + \sqrt{-3}}{2}]$ and $I$ is not prime to the conductor, meaning it is not invertible.)

\item[3]
    Complete the proof of Corollary 2, Theorem 1.

    The statement of the text leaves off with $\alpha$ being an algebraic integer if and only if $2r$ and $r^2 - ms^2$ are both integers, where $r, s \in \mathbb{Q}$.

    $2r$ being an integer requires that $r = \frac{a}{2}$, where $a$ is an integer.  Substituting $r = \frac{a}{2}$ into the second equation, we see that $a^2 - 4ms^2$ is an integer divisible by $4$.  In order for the quantity to be an integer, $s = \frac{b}{2}$, where $b$ is an integer.  Therefore $\alpha$ is an algebraic integer of the form $\frac{a + b\sqrt{m}}{2}$ if and only if $a^2 - mb^2 = 0 \mod 4$.

    We finish by considering $m \mod 4$ and seeing under which statements the given equation is solvable.  The key is that integer squares are either equivalent to 0 or 1 modulo 4.
    \begin{itemize}
        \item {\bf $m \equiv 1 \mod 4$}:  Let $a$ be even - then $a^2 \equiv 0 \mod 4$, and to satisfy the equality, $b^2 \equiv 0 \mod 4$ and so $b$ must also be even.  Similarly, if $a$ is odd, then $a^2 \equiv 1 \mod 4$ - to satisfy the equality, $b$ must also be odd.  Therefore $\alpha = \frac{a + b\sqrt{m}}{2}$ for all $a \equiv b\ (2)$ as required.
        \item {\bf $m \equiv 2, 3 \mod 4$}: For the equation to be solvable, both $a$ and $b$ must be equivalent to 0 or 2 modulo 4 (and so even), meaning $\alpha = c + d\sqrt{m}$ for $c, d \in \mathbb{Z}$ as required.
    \end{itemize}

\item[4]
    Suppose $a_0, \ldots, a_{n_1}$ are algebraic integers and $\alpha$ is a complex number satisfiying $\alpha^n + a_{n-1}\alpha^{n-1} + \cdots + a_1\alpha + a_0 = 0$.  Show the ring $\mathbb{Z}{[a_0, \ldots, a_{n-1}, \alpha]}$ has a finitely generated additive group.

    For each $a_i$ let $k_i$ be the degree of the algebraic integer $a_i$ over $\mathbb{Q}$: therefore for any power $k >= k_i$, it can be written as a linear combination of powers of $a_i$ less than $k_i$.  Additionally any power of $\alpha^k$ where $k \ge n$ can be written as a linear combination of powers of $\alpha$ multiplied by each of the $a_i$.  Therefore only a finite number of powers of $a_0^{m_0} \cdots a_n^{m_n} \alpha^{m}$ are needed; the $a_i$ terms are capped to be lower than $k_i$ and the $\alpha$ term is capped to be lower than $n$.

    Since $\alpha$ is a member of a subring of $\mathbb{C}$ that is finitely generated, $\alpha$ is therefore an algebraic integer.

\item[5]
    Let $f$ be a polynomial over $\mathbb{Z}_p$ where $p$ is a prime.  We prove $f(x^p) = (f(x))^p$ by induction on number of terms.

    If $f(x) = kx^{b}$ where $k \in \mathbb{Z}_p$, then $f(x^p) = kx^{pb} = k^p x^{bp} = (kx^{b})^p$ (since $k^p = k$ for all $k \in \mathbb{Z}_p$).

    Next, let $f(x) = g(x) + h(x)$ where $g(x)$ and $h(x)$ have fewer terms than $f(x)$.
    \begin{eqnarray*}
        f(x)^p          &=& (g(x) + h(x))^p \\
                        &=& g(x)^p + h(x)^p + \sum_{k = 1} \binom{p}{k} g(x)^{k} h(x)^{p - k} \\
                        &=& g(x)^p + h(x)^p \\
                        &=& g(x^p) + h(x^p) \text{ (using the inductive hypothesis)}\\
                        &=& f(x^p)
    \end{eqnarray*}
    This is the required result.

\item[6] If $f$ and $g$ are polynomials over a field $K$ and $f^2 \mid g$, then $g = f^2 h$.  Therefore $g' = f^2 h' + 2 h f f'$, so $f \mid g'$.

\item[7] Complete the proof of Corollary 2, Theorem 3.

Let $\phi_{k}$ be the automorphism of $\mathbb{Q}[\omega]$ sending $\omega$ to $\omega^k$.  Then $(\phi_{a} \circ \phi_{b}) (\omega) = (\omega^{a})^{b} = \omega^{ab} = \phi_{ab}$, giving the required result that composition of automorphisms corresponds to multiplication modulo $m$.

\item[8]
\begin{enumerate}
    \item[(a)] Let $\w = e^{2\pi i/p}$ where $p$ is an odd prime.
    Then \[ \text{disc}(\w) = \prod_{1 \le r < s \le n} (\alpha_r - \alpha_s)^2 = \pm p^{p - 2} \]
    Therefore \[ \Big\lvert \prod_{1 \le r < s \le n} (\alpha_r - \alpha_s)\ \Big\rvert = \sqrt{\pm p^{p - 2}} = p^{(p - 3) / 2} \sqrt{\pm p} \]

    Let $\z = e^{2\pi i /3}$.  Using the above we have the identity $(\z - \z^2) = \sqrt{-3}$.

    Let $\z = e^{2\pi i / 5}$.  Note $\z^4 = -(\z^3 + \z^2 + \z + 1)$.

    We expand the product: \[ (\z - \z^2)(\z - \z^3)(\z - z^4)(\z^2 - \z^3)(\z^2 - \z^4)(\z^3 - \z_4) = 10\z^3 + 10\z^2 + 1 \]

    Observing that this product is negative we flip the signs and divide by $5^{(5 - 3)/2} = 5$ to get the identity $\sqrt{5} = -2\z^3 - 2\z^2 - 1$.

    \item[(b)] The 8th cyclotomic polynomial is $x^4 + 1$, so the 8th cyclotomic field contains all the roots of this equation, which includes $\sqrt{i} = (1/\sqrt{2})(1 + i)$ and its complex conjugate $(1/\sqrt{2})(1 - i)$.  Thus the 8th cyclotomic field also contains their sum $2 / \sqrt{2} = \sqrt{2}$.

    \item[(c)] Let $m$ be a squarefree number.  Then $m$ can be written as $2^{i} q$ where $2 \nmid q$, and $i \in \{0, 1\}$.  We proceed by case analysis, showing for each that $\sqrt{m}$ is contained in the $d$th cyclotomic field, where $d = \text{disc}(\mathbb{A} \cap \mathbb{Q}[\sqrt{m}])$.

    $m = -1$: $\sqrt{-1}$ is contained in the 4th cyclotomic field which contains the complex unit $i$ ($d = -4$).

    $m = 2$: $\sqrt{2}$ is contained in the 8th cyclotomic field by part (b) ($d = 4\cdot 2 = 8$).

    $m = -2$: The 8th cyclotomic field contains $i$ (since it contains the 4th cyclotomic field as a subfield) so it contains $\sqrt{-2} = i\sqrt{2}$ ($d = 4\cdot -2 = -8$).

    $m = q$ where $q \equiv 1 \mod 4$: Because $q \equiv 1 \mod 4$, $q$ has an even number of prime factors $\equiv 3 \mod 4$, meaning that $\sqrt{q}$ must be contained in the $q$-th cyclotomic field ($d = q$ since $q\equiv 1\mod 4$).

    $m = q$ where $q \equiv 3 \mod 4$: The $4q$-th cyclotomic field contains the $q$-th cyclotomic field (containing $\sqrt{-q}$) and the 4th cyclotomic field (containing $\sqrt{-1}$) ($d = 4q$ since $q \equiv 3 \mod 4$), and so contains $\sqrt{q}$.

    $m = 2q$ where $q$ is a product of odd primes:  Here $d = 8q$.  By the above, $\sqrt{q}$ is contained in either the $q$-th or $4q$-th cyclotomic field, depending on its residue mod 4.  Thus $\sqrt{2q}$ is contained in the $8q$-th cyclotomic field.

    This shows every quadratic field $\mathbb{Q}[\sqrt{m}]$ is contained within the $d$-th cyclotomic field.
\end{enumerate}

\item[9] Let $\theta$ be a primitive $k$-th root of unity, i.e. $\theta = e^{2\pi i / k}$.  Let $\text{gcd}(k, m) = d$.  Using Euclid's extended algorithm we can find $u, v$ such that $uk + vm = d$.  Then we have
\[ \w^u \theta^v = e^{(2\pi i u)/m} e^{(2\pi i v) / k} = e^{2\pi i (uk + vm) / km} = e^{2\pi i d / km} = e^{2\pi i / r} \]
where $r = \text{lcm}(k, m)$ ($\text{lcm}(k, m) = km / \text{gcd}(k, m)$).

\item[10] Show if $m$ is even, $m \mid r$, and $\phi(r) \le \phi(m)$ then $r = m$.

If $m \mid r$ there is some $k$ such that $mk = r$.  Let $d = \gcd(k, m)$, so $r = mdj$ with $j$ satisfiying $\gcd(j, m) = 1$.  Therefore $\phi(r) = \phi(md)\phi(j)$.  Since $d \mid m$, $\phi(md) = d \cdot \phi(m)$, so \[ \phi(r) = d \cdot \phi(m)\phi(j) \le \phi(m) \]  The inequality forces $d = 1$ and $\phi(j) = 1$.  Because $2 \mid m \mid r$, $\phi(j) = 1$ implies $j = 1$.  Therefore $m = r$.

\end{enumerate}

\end{document}

