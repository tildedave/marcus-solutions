\documentclass{article}

\usepackage{fancyhdr}
\usepackage{amsmath}
\usepackage{amsthm}
\usepackage{amssymb}
\usepackage{amsfonts}
\usepackage{MnSymbol}

\newcommand{\w}[0]{\omega}
\newcommand{\z}[0]{\zeta}
\newcommand{\Q}[0]{\mathbb{Q}}
\newcommand{\R}[0]{\mathbb{R}}
\newcommand{\Z}[0]{\mathbb{Z}}
\newcommand{\p}[0]{\mathfrak{p}}
\newcommand{\q}[0]{\mathfrak{q}}
\newcommand{\zquad}[1]{\mathbb{Z}[\sqrt{#1}]}
\newcommand{\qext}[1]{\mathbb{Q}[#1]}
\newcommand{\zext}[1]{\mathbb{Z}[#1]}
\newcommand{\trace}[1]{\text{Tr}(#1)}
\newcommand{\norm}[0]{\text{N}}
\newcommand{\diff}[1]{\text{diff}\ #1}
\newcommand{\disc}[1]{\text{disc}(#1)}
\newcommand{\kernel}[0]{\text{kern}}
\newcommand{\modequiv}[3]{#1 \equiv #2\ (#3)}
\newcommand{\modnotequiv}[3]{#1 \not\equiv #2\ (#3)}
\newcommand{\gal}[2]{\text{Gal}(#1 / #2)}
\newcommand{\ringofintegers}[1]{\mathbb{A} \cap #1}
\newcommand{\legendre}[2]{\genfrac{(}{)}{}{}{#1}{#2}}

\newtheorem{theorem}{Theorem}[section]
\newtheorem{lemma}{Lemma}

\begin{document}

\section*{Chapter 2}

\begin{enumerate}

\item[1. (a)] Show every number field of degree 2 over $\mathbb{Q}$ is one of the quadratic fields.

Let $K$ be a number field of degree 2, and $f(x) = x^2 + px + q$ be its minimum polynomial over $\mathbb{Q}$.  Since $p, q \in \mathbb{Q}$ we can multiply through to clear the denominators and give us a polynomial $g(x) = ax^2 + bx + c$ over $\mathbb{Z}$ with the same roots as $f(x)$.  Therefore $K = \mathbb{Q}[\sqrt{b^2 - 4ac}]$ is a quadratic field for $m = b^2 - 4ac$.

\item[1. (b)] Suppose $K = \mathbb{Q}[\sqrt{m}]$ contains $\sqrt{n}$ for $n$ a squarefree integer.  Since $K$ has the basis $\{1, \sqrt{m}\}$, so $\sqrt{n} = p + q\sqrt{m}$ for $p, q \in \mathbb{Q}$. Therefore $n = p^2 + 2pq\sqrt{m} + q^2m$, so either $p = 0$ or $q = 0$.

If $p = 0$, then $\sqrt{n} = q\sqrt{m}$ and so $\sqrt{n} / \sqrt{m} = q$.  This can only happen if $q = 1$, meaning $m = n$.

If $q = 0$, then $\sqrt{n} = p$, which can only happen if $p$ is also an integer, contradicting $n$ squarefree.

Therefore the quadratic fields are each distinct.

\item[2.]
    Let $I$ be the ideal generated by $2$ and $1 + \sqrt{-3}$ in the ring $\mathbb{Z}[\sqrt{-3}]$.

    We have $I \neq (2)$ because $1 + \sqrt{-3}$ ($\in I$) does not have the form $2a + b\sqrt{-3}$ for $a, b \in \mathbb{Z}$.  The ideal $I^2$ is generated by $(4, 2 + 2\sqrt{-3}, -2 + 2\sqrt{-3})$.  The number $-2 + 2\sqrt{-3} = 2 + 2\sqrt{-3} - 4$ and so is redundant as a generator; therefore $I^2 = (4, 2 + 2\sqrt{-3}) = 2I$.

    Since $I^2 = 2I$, prime factorization of ideals in $\mathbb{Z}[\sqrt{-3}]$ must not hold; if we did then $I$ would be invertible, meaning it could be cancelled from the right-hand-side of each equality, giving us $I = (2)$ which is not true (from above).

    Suppose $P$ is a prime ideal of $\mathbb{Z}[\sqrt{-3}]$ containing $2$.  Then $4 \in P$ also.  Since $(1 + \sqrt{-3})(1 - \sqrt{-3}) = 4$ and $P$ is a prime ideal, one of $1 + \sqrt{-3}$ and $1 - \sqrt{-3}$ are also in $P$.  However, if $1 - \sqrt{-3} \in P$ then $1 + \sqrt{-3} \in P$ since $-1 \cdot (1 - \sqrt{-3}) + 2 = 1 + \sqrt{-3}$.  Therefore any prime ideal containing $(2)$ also contains $I$ and $I$ is the unique prime ideal that contains $(2)$.  Since $I$ cannot be expressed as a product of prime ideals, neither can $(2)$.

    (We should expect this; $\mathbb{Z}[\sqrt{-3}]$ is an order of conductor $2$ in $\mathbb{Z}[\frac{1 + \sqrt{-3}}{2}]$ and $I$ is not prime to the conductor, meaning it is not invertible.)

\item[3.]
    Complete the proof of Corollary 2, Theorem 1.

    The statement of the text leaves off with $\alpha$ being an algebraic integer if and only if $2r$ and $r^2 - ms^2$ are both integers, where $r, s \in \mathbb{Q}$.

    $2r$ being an integer requires that $r = \frac{a}{2}$, where $a$ is an integer.  Substituting $r = \frac{a}{2}$ into the second equation, we see that $a^2 - 4ms^2$ is an integer divisible by $4$.  In order for the quantity to be an integer, $s = \frac{b}{2}$, where $b$ is an integer.  Therefore $\alpha$ is an algebraic integer of the form $\frac{a + b\sqrt{m}}{2}$ if and only if $a^2 - mb^2 = 0 \mod 4$.

    We finish by considering $m \mod 4$ and seeing under which statements the given equation is solvable.  The key is that integer squares are either equivalent to 0 or 1 modulo 4.
    \begin{itemize}
        \item {\bf $\modequiv{m}{1}{4}$}:  Let $a$ be even - then $a^2 \equiv 0 \mod 4$, and to satisfy the equality, $b^2 \equiv 0 \mod 4$ and so $b$ must also be even.  Similarly, if $a$ is odd, then $a^2 \equiv 1 \mod 4$ - to satisfy the equality, $b$ must also be odd.  Therefore $\alpha = \frac{a + b\sqrt{m}}{2}$ for all $a \equiv b\ (2)$ as required.
        \item {\bf $m \equiv 2, 3 \mod 4$}: For the equation to be solvable, both $a$ and $b$ must be equivalent to 0 or 2 modulo 4 (and so even), meaning $\alpha = c + d\sqrt{m}$ for $c, d \in \mathbb{Z}$ as required.
    \end{itemize}

\item[4.]
    Suppose $a_0, \ldots, a_{n_1}$ are algebraic integers and $\alpha$ is a complex number satisfiying $\alpha^n + a_{n-1}\alpha^{n-1} + \cdots + a_1\alpha + a_0 = 0$.  Show the ring $\mathbb{Z}{[a_0, \ldots, a_{n-1}, \alpha]}$ has a finitely generated additive group.

    For each $a_i$ let $k_i$ be the degree of the algebraic integer $a_i$ over $\mathbb{Q}$: therefore for any power $k >= k_i$, it can be written as a linear combination of powers of $a_i$ less than $k_i$.  Additionally any power of $\alpha^k$ where $k \ge n$ can be written as a linear combination of powers of $\alpha$ multiplied by each of the $a_i$.  Therefore only a finite number of powers of $a_0^{m_0} \cdots a_n^{m_n} \alpha^{m}$ are needed; the $a_i$ terms are capped to be lower than $k_i$ and the $\alpha$ term is capped to be lower than $n$.

    Since $\alpha$ is a member of a subring of $\mathbb{C}$ that is finitely generated, $\alpha$ is therefore an algebraic integer.

\item[5.]
    Let $f$ be a polynomial over $\mathbb{Z}_p$ where $p$ is a prime.  We prove $f(x^p) = (f(x))^p$ by induction on number of terms.

    If $f(x) = kx^{b}$ where $k \in \mathbb{Z}_p$, then $f(x^p) = kx^{pb} = k^p x^{bp} = (kx^{b})^p$ (since $k^p = k$ for all $k \in \mathbb{Z}_p$).

    Next, let $f(x) = g(x) + h(x)$ where $g(x)$ and $h(x)$ have fewer terms than $f(x)$.
    \begin{eqnarray*}
        f(x)^p          &=& (g(x) + h(x))^p \\
                        &=& g(x)^p + h(x)^p + \sum_{k = 1} \binom{p}{k} g(x)^{k} h(x)^{p - k} \\
                        &=& g(x)^p + h(x)^p \\
                        &=& g(x^p) + h(x^p) \text{ (using the inductive hypothesis)}\\
                        &=& f(x^p)
    \end{eqnarray*}
    This is the required result.

\item[6.] If $f$ and $g$ are polynomials over a field $K$ and $f^2 \mid g$, then $g = f^2 h$.  Therefore $g' = f^2 h' + 2 h f f'$, so $f \mid g'$.

\item[7.] Complete the proof of Corollary 2, Theorem 3.

Let $\phi_{k}$ be the automorphism of $\mathbb{Q}[\omega]$ sending $\omega$ to $\omega^k$.  Then $(\phi_{a} \circ \phi_{b}) (\omega) = (\omega^{a})^{b} = \omega^{ab} = \phi_{ab}$, giving the required result that composition of automorphisms corresponds to multiplication modulo $m$.

\item[8. (a)] Let $\w = e^{2\pi i/p}$ where $p$ is an odd prime.
Then \[ \text{disc}(\w) = \prod_{1 \le r < s \le n} (\alpha_r - \alpha_s)^2 = \pm p^{p - 2} \]
Therefore \[ \Big\lvert \prod_{1 \le r < s \le n} (\alpha_r - \alpha_s)\ \Big\rvert = \sqrt{\pm p^{p - 2}} = p^{(p - 3) / 2} \sqrt{\pm p} \]

Let $\z = e^{2\pi i /3}$.  Using the above we have the identity $(\z - \z^2) = \sqrt{-3}$.

Let $\z = e^{2\pi i / 5}$.  Note $\z^4 = -(\z^3 + \z^2 + \z + 1)$.

We expand the product: \[ (\z - \z^2)(\z - \z^3)(\z - z^4)(\z^2 - \z^3)(\z^2 - \z^4)(\z^3 - \z_4) = 10\z^3 + 10\z^2 + 1 \]

Observing that this product is negative we flip the signs and divide by $5^{(5 - 3)/2} = 5$ to get the identity $\sqrt{5} = -2\z^3 - 2\z^2 - 1$.

\item[8. (b)] The 8th cyclotomic polynomial is $x^4 + 1$, so the 8th cyclotomic field contains all the roots of this equation, which includes $\sqrt{i} = (1/\sqrt{2})(1 + i)$ and its complex conjugate $(1/\sqrt{2})(1 - i)$.  Thus the 8th cyclotomic field also contains their sum $2 / \sqrt{2} = \sqrt{2}$.

\item[8. (c)] Let $m$ be a squarefree number.  Then $m$ can be written as $2^{i} q$ where $2 \nmid q$, and $i \in \{0, 1\}$.  We proceed by case analysis, showing for each that $\sqrt{m}$ is contained in the $d$th cyclotomic field, where $d = \text{disc}(\mathbb{A} \cap \mathbb{Q}[\sqrt{m}])$.

$m = -1$: $\sqrt{-1}$ is contained in the 4th cyclotomic field which contains the complex unit $i$ ($d = -4$).

$m = 2$: $\sqrt{2}$ is contained in the 8th cyclotomic field by part (b) ($d = 4\cdot 2 = 8$).

$m = -2$: The 8th cyclotomic field contains $i$ (since it contains the 4th cyclotomic field as a subfield) so it contains $\sqrt{-2} = i\sqrt{2}$ ($d = 4\cdot -2 = -8$).

$m = q$ where $q \equiv 1 \mod 4$: Because $q \equiv 1 \mod 4$, $q$ has an even number of prime factors $\equiv 3 \mod 4$, meaning that $\sqrt{q}$ must be contained in the $q$-th cyclotomic field ($d = q$ since $q\equiv 1\mod 4$).

$m = q$ where $q \equiv 3 \mod 4$: The $4q$-th cyclotomic field contains the $q$-th cyclotomic field (containing $\sqrt{-q}$) and the 4th cyclotomic field (containing $\sqrt{-1}$) ($d = 4q$ since $q \equiv 3 \mod 4$), and so contains $\sqrt{q}$.

$m = 2q$ where $q$ is a product of odd primes:  Here $d = 8q$.  By the above, $\sqrt{q}$ is contained in either the $q$-th or $4q$-th cyclotomic field, depending on its residue mod 4.  Thus $\sqrt{2q}$ is contained in the $8q$-th cyclotomic field.

This shows every quadratic field $\mathbb{Q}[\sqrt{m}]$ is contained within the $d$-th cyclotomic field.

\item[9.] Let $\theta$ be a primitive $k$-th root of unity, i.e. $\theta = e^{2\pi i / k}$.  Let $\text{gcd}(k, m) = d$.  Using Euclid's extended algorithm we can find $u, v$ such that $uk + vm = d$.  Then we have
\[ \w^u \theta^v = e^{(2\pi i u)/m} e^{(2\pi i v) / k} = e^{2\pi i (uk + vm) / km} = e^{2\pi i d / km} = e^{2\pi i / r} \]
where $r = \text{lcm}(k, m)$ ($\text{lcm}(k, m) = km / \text{gcd}(k, m)$).

\item[10.] Show if $m$ is even, $m \mid r$, and $\phi(r) \le \phi(m)$ then $r = m$.

If $m \mid r$ there is some $k$ such that $mk = r$.  Let $d = \gcd(k, m)$, so $r = mdj$ with $j$ satisfiying $\gcd(j, m) = 1$.  Therefore $\phi(r) = \phi(md)\phi(j)$.  Since $d \mid m$, $\phi(md) = d \cdot \phi(m)$, so \[ \phi(r) = d \cdot \phi(m)\phi(j) \le \phi(m) \]  The inequality forces $d = 1$ and $\phi(j) = 1$.  Because $2 \mid m \mid r$, $\phi(j) = 1$ implies $j = 1$.  Therefore $m = r$.

\item[11. (a)] Suppose all the roots to a monic polynomial $f$ have absolute value $1$.  Show that the coefficient of $x^r$ has absolute value $\le \binom{n}{r}$, where $n$ is the degree of $f$ and $\binom{n}{r}$ is the binomial coefficient.

Factor $f$ as $f = (x - \alpha_0) \cdots (x - \alpha_n)$.  Re-expanding $f$ we see that the coefficient of $x^r$ is equal to $\sum_{S \subseteq \{0, \ldots, n\}, |S| = r} x^r \prod_{i \in S} \alpha_i$.  By assumption $|\alpha_i| = 1$ for all $i$, so $|\prod_{i \in S} \alpha_i| = 1$.  There are $\binom{n}{r}$ of these subsets of $S$.

Using the identity $|a + b| \le |a| + |b|$ we have:
\begin{eqnarray*}
    \left|\sum_{S \subseteq \{0, \ldots, n\}, |S| = r} \prod_{i \in S} \alpha_i \right| &\le& \sum_{S \subseteq \{0, \ldots, n\}, |S| = r} | \prod_{i \in S} \alpha_i | \\
    &\le& \sum_{S \subseteq \{0, \ldots, n\}, |S| = r} 1 \\
    &\le& \binom{n}{r}
\end{eqnarray*}

\item[11. (b)] We will consider all monic polynomials $f$ of degree $n$ and show that only a finite number of them can have a root $\alpha$ all of whose conjugates have absolute value 1.

By Theorem 1, if $\alpha$ is an algebraic integer, than the coefficients of $f$ are integers.  By (b), the absolute value of the coefficients of $f$ are bounded above $\binom{n}{r}$, therefore there are at most $2\binom{n}{r}$ choices for each coefficient beyond the $x^n$th term.  The constant term of the polynomial must be 1 (since $\alpha$ has absolute value 1) and the first term of the polynomial must also be 1 (since $f$ is monic). This gives an upper bound of $\sum_{r=1}^{n - 1} 2\binom{n}{r} = 2(2^n - 2) = 4(2^{n-1} - 1)$ on the number of algebraic integers satisfying the given condition.

\item[11. (c)]{(TODO)}

\item[12. (a)] Let $u$ be a unit in $\Z[\w]$, where $\w = e^{2\pi i / p}$.  Show $u / \overline{u}$ is a root of 1.

The field $\Q[\w]$ has Galois group $\simeq \Z_p^{\times}$, which has cardinality $p - 1$ and so has an element of order 2 (complex conjugation).  Therefore $u$ has $p - 1$ conjugates, which consist of $(p - 1) / 2$ elements along with their complex conjugates.  Enumerate the conjugates of $u$ as $a_1, \ldots, a_n, \overline{a_1}, \ldots, \overline{a_n}$.

Therefore, the conjugates of $u / \overline{u}$ have the form $a_i / \overline{a_i}$ or $\overline{a_i} / a_i$.  Multiplying over all conjugates of $u / \overline{u}$, we have $\prod_{i = 0}^{n} a_i / \overline{a_i} \cdot \prod_{i = 0}^{n} \overline{a_i} / a_i = 1$, and so $u / \overline{u}$ and all its conjugates have absolute value 1.  By 11 (c), $u / \overline{u}$ is then a root of 1, and so has form $\pm \omega^{k}$.

\item[12. (b)] Suppose $u / \overline{u} = -\omega^{k}$.  We derive a contradiction.  Raising both sides to the $p$-th power we have $u^p / \overline{u^p} = -(\omega^{k})^p = -(\omega^{p})^k = -1$, and so $u^p = -\overline{u^p}$.  By exercise 1.25, $u^p \equiv a\ (p)$ for some $a \in \Z$.  Applying exercise 1.23, we see $\overline{u^p} \equiv \overline{a} = a\ (p)$, and so $a \equiv -a\ (p)$.  There $a$ must be 0, and $u^p \equiv 0\ (p)$, so $p$ divides $u^p$.  This contradicts $u^p$ being a unit, since if $p$ divided $u^p$, p would also divide the absolute value of $u^p$, which is 1.  Therefore $u / \overline{u} = \omega^{k}$.

\item[13.] Show that 1 and -1 are the only units in the ring $A \cap \Q[\sqrt{m}]$, $m$ squarefree and $m < 0, m \neq -1, -3$.  What if $m = -1, -3$?

Let $u$ be a unit in $A \cap \Q[\sqrt{m}]$.  Then $u = a + b\sqrt{m}$ where $p, q \in A \cap \Q[sqrt{m}]$. Since $N(u) = 1$, then $(a + b\sqrt{m})(a - b\sqrt{m}) = a^2 - b^2 m = 1$.  We proceed by cases on whether $\modequiv{m}{1}{4}$.

If $m \not\equiv 1 \mod 4$, then $a$ and $b$ must be integers and so $a^2 - b^2 m = 1$ can only be satisfied if one of the terms is 1 and the other is 0.  If $a^2 = 1$, then $b^2 m = 0$.  This corresponds to the units 1 and -1 in $A \cap \Q[\sqrt{m}]$.  If $-b^2 m = 1$, then $b^2 m = -1$ and so $m = -1$.  This corresponds to the units $i$ and $-i$ in $A \cap \Q[\sqrt{-1}]$.

If $\modequiv{m}{1}{4}$ then let $a = r / 2$ and $b = s / 2$.  Therefore $r^2 - s^2 m = 4$.  Since $m$ is negative, both $r^2$ and $-s^2 m$ must be positive.  $r^2$ must be either $0$, $1$, or $4$.

If $r^2$ is 0 then $-s^2 m = 4$, so $s^2 m = -4$, forcing $m = -1$ which is not $\equiv \mod 4$.  (We have considered this case already.)

If $r^2$ is 1 then $-s^2 m = 3$ so $s^2 m = -3$ and $m = -3$, $s = \pm 1$.  This corresponds to the unit $\pm \frac{1}{2} \pm \frac{\sqrt{-3}}{2}$ in the ring $A \cap \Q[\sqrt{-3}]$.

If $r^2$ is 4 then $- s^2 m = 0$, which corresponds to the unit $\pm 1$ in the ring $A \cap \Q[\sqrt{m}]$.

\item[14.] Show that $1 + \sqrt{2}$ is a unit in $\mathbb{Z}[\sqrt{2}]$, but not a root of 1.

$1 + \sqrt{2}$ is a unit, as $-(1 - \sqrt{2})$ is its inverse: \[ -(1 + \sqrt{2})(1 - \sqrt{2}) = -1 + (\sqrt{2})^2 = 1 \]

If $1 + \sqrt{2}$ were a root of 1, we would have $(1 + \sqrt{2})^k = 1$ for some $k$.  However by the Binomial Theorem, $(1 + \sqrt{2})^k = \sum^{k}_{i = 0} \binom{k}{i} (\sqrt{2})^i$, which will always contains a term $\sqrt{2}$ multiplied by a positive number.  Therefore $1 + \sqrt{2}$ is not a root of 1.

Let $(1 + \sqrt{2})^k = a + b\sqrt{2}$.  The inverse of this term is \[((1 + \sqrt{2})^k)^{-1} = ((1 + \sqrt{2})^{-1})^{k} = (-1)^k (1 - \sqrt{2})^k = (-1)^k (a - b\sqrt{2})^k \]

Therefore, $(a + b\sqrt{2})^k \cdot (a - b\sqrt{2})^k = \pm 1$ and so the powers of $1 + \sqrt{2}$ give an infinite number of $a$, $b$ such that $a^2 - 2b^2 = \pm 1$.

\item[15. (a)] Let $a + b\sqrt{-5}$ be an element of $\zquad{-5}$.  Then the norm of $a + b\sqrt{-5}$ is $(a + b\sqrt{-5})(a - b\sqrt{-5}) = a^2 + 5b^2$, where $a, b \in \mathbb{Z}$.  Since there are no integer solutions $a, b$ such that $a^2 + 5b^2 = 2$ or $a^2 + 5b^2 = 3$, there can be no element of $\zquad{-5}$ with a norm of $2$ or $3$.
\item[15. (b)] In $\zquad{-5}$, $6 = 2 \cdot 3 = (1 + \sqrt{-5})(1 - \sqrt{-5})$.  If unique factorization held in $\zquad{-5}$, there would be elements $a, b, c, d \in \zquad{-5}$ such that $a \cdot b = 2$, $c \cdot d = 3$, $a \cdot d  =  1 + \sqrt{-5}$, $b \cdot c = 1 - \sqrt{-5}$.
    However by (a), 2 and 3 are irreducible in $\zquad{-5}$, meaning they are irreducible elements, and so no $a, b, c, d$ can exist.

\item[16.]
We argue in the style of K. Conrad: Trace and Norm, Section 4.  Suppose $\sqrt{3} \in \qext{\alpha}$ where $\alpha = \sqrt[4]{2}$; therefore $\sqrt{3} = a + b\alpha + c\alpha^2 + d\alpha^3$.  We have the following traces:
\begin{eqnarray*}
    \trace{\sqrt{3}} &=& \sqrt{3} - \sqrt{3} = 0 \\
    \trace{\alpha} &=& \alpha - \alpha + i\alpha - i\alpha = 0\\
    \trace{\alpha^2} &=& \alpha^2 - \alpha^2 + i\alpha^2 - i\alpha^2 = 0\\
    \trace{\alpha^3} &=& \alpha^3 - \alpha^3 + i\alpha^3 - i\alpha^3 = 0\\
\end{eqnarray*}
Since $\sqrt{3} = a + b\alpha + c\alpha^2 + d\alpha^3$,
\begin{eqnarray*}
    \trace{\sqrt{3}} &=& \trace{a + b\alpha + c\alpha^2 + d\alpha^3} \\
    0 &=& a\trace{1} + b\trace{\alpha} + c\trace{\alpha^2} + d\trace{\alpha^3} \\
    0 &=& 4a \\
\end{eqnarray*}
Therefore $a = 0$, and we have $\sqrt{3} = b\alpha + c\alpha^2 + d\alpha^3$.  We have $\trace{\sqrt{3}\alpha} = \trace{\sqrt[4]{9/2}} = \sqrt[4]{9/2} - \sqrt[4]{9/2} + i\sqrt[4]{9/2} - i\sqrt[4]{9/2} = 0$, so $0 = b\trace{1} + c\trace{\alpha} + d\trace{\alpha}^2 = 4b$ and so $b = 0$.

Similarly $\trace{\sqrt{3}/\alpha^2} = \trace{\sqrt{3/2}} = 0$, and so $c = 0$.

From eliminating the coefficients $a, b, c$, we have $d\sqrt[4]{8} = \sqrt{3}$ and so $3 = d^2\sqrt{8} = 2d^2\sqrt{2}$.  Therefore $\sqrt{2}$ is expressible as a rational number $3/d^2$, a contradiction.  Therefore $\sqrt{3} \not\in \qext{\alpha}$.

(Where would this argument break down for $\sqrt{2}$?  $\sqrt{2} = \alpha^2$ so $\sqrt{2} / \alpha^2 = 1$ and so we would conclude that $c = 1$ rather than $c = 0$.)

\item[17 - TODO]

\item[18 - TODO]

\item[19 - TODO]

\item[20.] Write $f(x) = (x - \alpha)g(x)$.  By the chain rule $f'(x) = (x - \alpha)g'(x) + g(x)$, so $f'(\alpha) = g(\alpha) = \prod_{\beta \neq \alpha} (\alpha - \beta)$.

\item[21.] Let $f(x) = g(x) h(x)$, where $g(x)$ is the minimum polynomial of $\alpha$ over $\mathbb{Z}$.  Then $f'(x) = g'(x)h(x) + g(x) h'(x)$ and $f'(\alpha) = g'(\alpha)h(\alpha)$.  We have \[ \norm(f'(\alpha)) = \norm(g'(\alpha))\norm(h(\alpha)) \].  By Theorem 8, $\norm(g'(\alpha)) = \pm \disc{\alpha}$, so \[ \norm(f'(\alpha)) = \pm \disc{\alpha}\norm(h(\alpha)) \]
Therefore $\disc{\alpha}$ divides $\norm(f'(\alpha))$ as required.

\item[23. (c)] Let $\{ \alpha_1, \ldots, \alpha_n \}$ be an integral basis for $K$ ($n = [K : \Q$) and let $\{ \beta_1, \ldots, \beta_m \}$ be an integral basis for $L$ ($m = [L : \Q]$).  Therefore \[ \{ \alpha_i \beta_j \ |\ 1 \le i \le n, 1 \le j \le m \] is an integral basis for $KL$.

We have the tower of field extensions $KL : K : \Q$ where $[KL : K] = m$, $[K : \Q] = n$.  By the formula established in (b), \[ \disc{\alpha_i \beta_j} = (\disc{\alpha_i})^m N^{K}_{\Q} \disc{\beta_j} = (\text{disc}\ R)^m (\text{disc}\ S)^n \]
Because $\text{disc}\ S$ is an integer, its norm is the degree of $K$ over $\Q$.

\item[24]  Let $G$ be a free abelian group of rank $n$ and let $H$ be a subgroup.  Take $G = \Z \oplus \cdots \oplus \Z$.  We show by induction that $H$ is a free abelian group of rank $\leq n$.

{\bf First prove the result for $n = 1$.}

If $G$ is a free abelian group of rank 1, $G = \Z$.  If $H$ is a subgroup of $G$ then $H$ must have a least non-negative element, call it $m$.  Then $H$ is generated by $m$ (all subgroups of $\Z$ are generated by a single element).

Next, we assume the result holds for $n - 1$, and define $\pi : G \to \Z$ the projection of $G$ onto the first factor.  Let $K$ denote the kernel of $\pi$.

{\bf (a): Show that $H \cap K$ is a free abelian group of rank $\le n - 1$.}

Let $\iota$ be the map that drops the first factor from $G$; as $K$ is a subgroup of $G$, then $\iota(H \cap K)$ must be a subgroup of $\iota(G)$.  $\iota(G)$ is a free abelian group of rank $n - 1$, and so applying the inductive hypothesis, we see $\iota(H \cap K)$ = $0 \oplus (H \cap K)$ is a free abelian group of order $n -1$.

{\bf (b): The image $\pi(H) \subset \Z$ is either $\{0\}$ or infinite cyclic.  If it is 0, then $H = H \cap K$.  Otherwise let $h \in \pi(H)$ be a generator of $\pi(H)$.  Show $H$ is the direct sum of its subgroups $\Z h$ and $K \cap H$.}

Let $h$ be as in the problem statement.  Let $a \in H$.  We will show $a$ is a member of $\Z h \oplus (K \cap H)$.  If $\pi(a) = 0$, then $a \in H \cap K$ and so $a$ is a member of the required group.  Otherwise $\pi(a) = m\pi(h)$ for some integer $m$ and so $mh - a \in K \cap H$ (a free abelian group of rank $\le n - 1$).  Therefore $a$ is the direct sum of $mh \in \Z h$ and the components of $mh - a$.  Since $a$ was chosen arbitrarily, $H = \Z h \oplus (K \cap H)$.

\item[25.] Let $\alpha$ be an algebraic number, so there is some $f \in \Q[x]$ such that $f(\alpha) = 0$.  We convert this polynomial into a (non-monic) $g \in \Z[x]$ by through multiplying by the GCD $m$ for all of the denominators in the coefficients of $f$.  Then $g = a_n x^n + a_{n-1} x^{n-1} + \cdots + a_0$ and $g(\alpha) = 0$.  Multiplying through by $a_n ^ {n - 1}$ gives the relationship $(a_n \alpha)^n + a_{n-1} a_n^{n- 1} \alpha^{n - 1} + \cdots + a_n^{n-1} a_0 = 0$.  This is a monic polynomial with integer coefficients, so $m a_n^{n} \alpha$ is an algebraic integer.

Given any finite set of algebraic numbers, $\{\alpha_0, \ldots \alpha_n \}$ let $m_i$ be such that $m_i \alpha_i$ is an algebraic integer.  Therefore taking $M$ to be the least common multiple of each $m_i$ gives us a number $M$ such that each $M\alpha_i$ is an algebraic integer.

\item[26.] The proof that two sets that generate the same subgroup have the same discriminant is the same as that of Theorem 11: as $\{\beta_1, \ldots, \beta_n\}$ and $\gamma_1, \ldots, \gamma_n\}$ generate the same additive subgroup, we can write the $\gamma_i$ in terms of the $\beta_i$ through an matrix $M$ with entries in $\Z$, and vice versa.  This shows that the translate matrices must have determinant 1, so the discriminants are equal.

\item[27.] Let $G$ and $H$ be two free abelian subgroups of rank $n$ in $K$, with $H \subset G$.
\item[27. (a)] Show $G / H$ is a finite group.

Since $G$ and $H$ are free abelian subgroups of rank $n$, $G \simeq \Z \oplus \cdots \oplus \Z$ and since $H$ is a subgroup of $G$, then $H \simeq I_1 \oplus \cdots \oplus I_n$, where each $I_i \subseteq \Z$ is an additive subgroup of $\Z$.  Each $\Z / I_i$ is finite, having cardinality equal to the generating element of $I_i$.  Therefore $G/H$ is finite, having cardinality $\prod_{i = 0}^{n} |\Z/I_i|$.

\item[27. (b)]  The well-known finite structure theorem for abelian groups says $G / H$ is a direct sum of at most $n$ cyclic groups.  Use this to show that $G$ has a generating set $\beta_1, \ldots, \beta_n$ such that for appropriate integers $d_i$, $d_1\beta_1, \ldots, d_n\beta_n$ is a generating set for $H$.

Let $\beta_i$ be 1 projected to the $i$th-factor and 0 elsewhere.  Then the set of $\{\beta_i\}$ generate $G$.  Let $d_i$ be the minimum element of $I_i$, an additive subgroup of $\Z$: we show $\{d_i\beta_i\}$ generates $H$.  Take $a \in H$, and let $\iota_i(a)$ be the $i$th factor of $a$, so $\iota_i(a) \in I_i$.  By choice of $d_i$, $\iota_i(a) = d_i m$ for some integer $m$, and $a = \iota_1(a) \oplus \cdots \oplus \iota_n(a) = d_1 \beta_1 + \cdots + d_n \beta_n$.  Since $a$ was chosen arbitrarily, the $\{d_i\beta_i\}$ generates $H$.

\item[27. (c)]  $\disc{H} = \disc{d_1\beta_1, \ldots, d_n\beta_n}$: by Exercise 3.18 (a), \[\disc{H} = (d_1 \cdots d_n)^2 \disc{\beta_1, \ldots, \beta_n} = |G/H|^2 \disc{G} \]

\item[27. (d)] Show that if $\alpha_1, \ldots, \alpha_n \in R = \ringofintegers{K}$, then they form an integral basis iff $\disc{\alpha_1, \ldots, \alpha_n} = \disc{R}$.

Let $H$ be the additive subgroup formed by $\alpha_1, \ldots, \alpha_n$.  By (c), we have $\disc{H} = |R/H|^2 \disc{R}$.  Therefore $\disc{R} = \disc{G}$ iff $|G/H|^2 = 1$, which is the same as saying that there is $b \in G$ such that $b \not\in H$.  Therefore $\disc{\alpha_1, \ldots, \alpha_n} = \disc{R}$ if and only if they form an integral basis for $R$.

\item[27. (e)] Show that if $\alpha_1, \ldots, \alpha_n \in R = \ringofintegers{K}$ and $\disc{\alpha_1, \ldots, \alpha_n}$ is squarefree, then the $\alpha_i$ form an integral basis for $R$.

If $\disc{H}$ is squarefree then $|R / H| = 1$ which implies that $\disc{H} = \disc{R}$.  By (d) the $\alpha_i$ form an integral basis for $R$.

\item [28. (a)] Taking the derivative of the polynomial, we have $f'(x) = 3x^2 + a$.  We then have:
\begin{eqnarray*}
    f'(\alpha) &=& 3\alpha^2 + a \\
    \alpha f'(\alpha) &=& 3\alpha^3 + a\alpha \\
    \alpha f'(\alpha) &=& -3(a\alpha + b) + a\alpha \\
    \alpha f'(\alpha) &=& -2a\alpha - 3b \\
    f'(\alpha) &=& -(2a\alpha + 3b) / \alpha \\
\end{eqnarray*}

\item [28. (b)] It is straightforward that $2a\alpha + 3b$ is a root of the polynomial $g(x) = (\frac{x - 3b}{2a})^3 + a(\frac{x - 3b}{2a}) + b$.  To calculate the norm of $2a\alpha + 3b$ over $\Q[\alpha]$, we thus divide the zero coefficient of $g(x)$ by negative the initial coefficient of $g(x)$ (negative since $n = 3$ is odd):
\[ -(2a)^3\left(\frac{(-3b)^3}{(2a)^3} - \frac{3b}{2} + b\right) \]
Reducing terms gives us
\[ \norm(2a\alpha + 3b) = (3b)^3 + (2^2)a^3b = 27b^3 + 4a^3b \]

\item[28. (c)] By Theorem 8, $\disc{a} = -\norm(f'(\alpha))$ (the negative sign holds since $n = 3 \not\equiv 0,1 \ (4)$, ).

Note that given the factoring of $f(x)$ into $(x - \alpha_1)(x - \alpha_2)(x - \alpha_3)$, $(-1)\alpha_1\alpha_2\alpha_3 = -\norm(\alpha) = b$, $\norm(\alpha) = -b$.

We now compute the discriminant of $\alpha$:
\begin{eqnarray*}
    \disc{\alpha} &=& -\norm(f'(\alpha)) \\
                    &=& -\norm(-(2a\alpha + 3b)/\alpha) \\
                    &=& \frac{27b^3 + 4a^3b}{-b} \\
                    &=& -(27b^2 + 4a^3)
\end{eqnarray*}
This is the required result.

\item[28. (d)] If $\alpha^3 = \alpha + 1$, then $a = -1$ and $b = -1$.  By (c), $\disc{\alpha} = -27 - 4 = -31$, which is squarefree.  By 27 (c) the powers of $\alpha$ thus form an integral basis for $\mathbb{A} \cap \Q[\alpha]$.

Similarly if $a = 1$ and $b = -1$, then $\disc{\alpha} = -27 + 4 = -23$ (squarefree) and so again by 27 (c) the powers of $\alpha$ form an integral basis for $\mathbb{A} \cap \Q[\alpha]$.

\item[29.] Let $\Q[\sqrt{m}, \sqrt{n}]$, where $(m, n) = 1$.  Find an integral basis and the discriminant of this basis for (a): the case where $m, n \equiv 1\ (4)$ and (b) where $m \equiv 1\ (4)$, $n \not\equiv 1\ (4)$.

For both given scenarios, the ring of integers is a linear combination of the ring of integers of $\Q[\sqrt{m}]$ and $\Q[\sqrt{n}]$, and so Theorem 12, Corollary 1 applies, and an integral basis can be found as a combination of the bases of the individual rings.

\item [29. (a)] $m, n \equiv 1\ (4)$: The corresponding rings of integers for $\Q[\sqrt{m}]$ and $\Q[\sqrt{n}]$ are $\Z[(1 + \sqrt{m})/2]$ and $\Z[(1 + \sqrt{n})/2)]$ with discriminants $m$ and $n$.  By assumption, these discriminants are relatively prime, so Theorem 12, Corollary 1 applies.  The field $\Q[\sqrt{m}, \sqrt{n}]$ thus has an integral basis $\{ 1, (\sqrt{m} + 1)/2, (\sqrt{n} + 1)/2, (1 + \sqrt{m} + \sqrt{n} + \sqrt{nm})/4 \}$.  By Exercise 23 (c), the discriminant for this basis is $m^2 n^2$ .

\item [29. (b)] The rings of integers for $\Q[\sqrt{m}]$ and $\Q[\sqrt{n}]$ are $\Z[(1 +\sqrt{m})/2]$ and $\Z[\sqrt{n}]$, with discriminants $m$ and $4n$.  Since $m$ was assumed to be squarefree, $(m, 4n) = 1$, so Theorem 12, Corollary 1 applies again.  The field $\Q[\sqrt{m}, \sqrt{n}]$ thus has an integral basis $\{ 1, (\sqrt{m} + 1)/2, \sqrt{n}, (\sqrt{mn} + \sqrt{n})/2 \}$.  By Exercise 23 (c), the discriminant for this basis is $m^2 (4n)^2 = 16m^2 n^2$.

\item[30.] Let $f$ be the monic irreducible polynomial for $\alpha$ over $\Z$ and for each $g \in \Z[x]$, let $\overline{g}$ denote the polynomial in $\Z_3[x]$ obtained by reducing the coefficients mod 3.

\item[30. (a)] Show that $g(\alpha)$ is divisible by 3 in $\Z[\alpha]$ if and only if $\overline{g}$ is divisible by $\overline{f}$ in $\Z_3[x]$.

Suppose $g(\alpha)$ is divisible by 3.  Then $g(\alpha) = 3m$ for some $m$ and so $(g - 3m)(\alpha) = 0$.  Since this is a polynomial in $\alpha$ and $f$ is the minimum polynomial, $f \mid g - 3m$.  Therefore $\overline{f} \mid \overline{g - 3m} = \overline{g}$.

If $\overline{g}$ is divisible by $\overline{f}$ in $\Z_3[x]$, then $\overline{g} = \overline{f}\overline{h}$ for some $h \in \Z[x]$, and so $g = (f + 3j)h$ in $\Z[x]$ for some polynomial $j(x) \in \Z[x]$.  So $g(\alpha) = 3j(\alpha)h(\alpha)$ and $g(\alpha)$ is divisible by 3.

\item[30. (b)] Consider the four algebraic integers:
\begin{eqnarray*}
    \alpha_1 &=& (1 + \sqrt{7})(1 + \sqrt{10}) \\
    \alpha_2 &=& (1 + \sqrt{7})(1 - \sqrt{10}) \\
    \alpha_3 &=& (1 - \sqrt{7})(1 + \sqrt{10}) \\
    \alpha_4 &=& (1 - \sqrt{7})(1 - \sqrt{10}) \\
\end{eqnarray*}

The conjugates of each $\alpha_i$ are the other $\alpha_j$, and each product $\alpha_i \alpha_j$ is divisible by 3: $\alpha_1 \alpha_3$, $\alpha_2 \alpha_3$, $\alpha_1 \alpha_4$, and $\alpha_2 \alpha_4$ are divisible by $-6$, and $\alpha_1 \alpha_2$, $\alpha_1 \alpha_4$, $\alpha_2 \alpha_3$, and $\alpha_3 \alpha_4$ are divisible by $-9$.

We show that $\alpha_i^n / 3$ is not an algebraic integer by considering its trace: $\trace{\alpha_i^n / 3} = \trace{\alpha_i^n} / 3$, so we compute $\trace{\alpha_i^n}$.  The conjugates of $\alpha_i^n$ are each of the other $\alpha_j^n$, so $\trace{\alpha_i^n} = \alpha_1^n + \alpha_2^n + \alpha_3^n + \alpha_4^n$.  Modulo 3, $(\alpha_1 + \alpha_2 + \alpha_3 + \alpha_4)^n \equiv \alpha_1^n + \alpha_2^n + \alpha_3^n + \alpha_4^n$ because any of the monomials with any nonzero powers is divisible by 3 and so cancel out mod 3.  However $(\alpha_1 + \alpha_2 + \alpha_3 + \alpha_4)^n = 1^n = 1$.  Since each $\alpha_i$ is conjugate to each of the $\alpha_j$, their traces must be identical.

Therefore the trace of $\alpha_i^n$ is an integer $\equiv 1\ (3)$, and so $\trace{\alpha_i^n / 3}$ cannot be an integer, and so by Corollary 2 to Theorem 4, $\alpha_i^n / 3$ must not be an algebraic integer.

\item[30. (c)] Let $\alpha_i$ from (b) be defined by $f_i(\alpha)$ (for any fixed $\alpha$).  Because $\alpha_i \alpha_j$ is divisible by 3, by (a), $\overline{f} \divides \overline{f_i}\overline{f_j}$.  However, $\overline{f} \not \divides \overline{f_i}^n$ for any power of $n$ (or else 3 would $\overline{f_i}^n$ which is not the case by (b)), so $\overline{f_i}\overline{f_j} \ne \overline{f_i}^n$ for any $n$.  Therefore, since $\Z_3[x]$ is a UFD, $\overline{f}$ has an irreducible factor that does not divide $\overline{f_i}$ but does divide $\overline{f_j}$ for all $j \neq i$.

\item[30. (d)] The result of (c) is that $\overline{f}$ has at least 4 irreducible factors in $\Z_3[x]$.  However, $\overline{f}$ is of degree at most 4, since $\alpha \in \Q[\sqrt{7}, \sqrt{10}]$.  For there to be at least 4 irreducible factors of $\overline{f}$ it would imply each are of degree 1, but there are only 3 monic polynomials of degree 1 in $\Z_3[x]$: $x$, $x - 1$, $x - 2$.  Therefore $\mathbb{A} \cap \Q[\sqrt{7}, \sqrt{10}] \neq \Z[\alpha]$ for any $\alpha$.

\item[31.] Show that $\frac{\sqrt{3} + \sqrt{7}}{2}$ is an algebraic integer.

$\frac{\sqrt{3} + \sqrt{7}}{2}$ is the root of the degree 4 polynomial $f(x) = x^4 - 5x^2 + 1$.  This shows that the intersection of the ring of integers $\Z[\sqrt{3}]$ and $\Z[\sqrt{7}]$ is not $\Z[\sqrt{3}, \sqrt{7}]$; neither original ring contains fractional elements.  (Their discriminants are 12 and 28 respectively, sharing a factor of 4.)

\item[32.] The fields $\Q[\sqrt[3]{2}]$ and $\Q[\w\sqrt[3]{2}]$ where $\w = e^{2\pi i/3}$ both have degree 3 over $\Q$, but their composition $\Q[\w, \sqrt[3]{2}]$ has degree 6 over $\Q$.

\item[33.] Let $\w = e^{2\pi i / m}$, where $m \ge 3$.  We know that $N(\w) = \pm 1$ because $\w$ is a unit.  Show the $+$ sign holds.

Write $e^{2\pi i k /m}$ as $\w_k$.  The conjugates of $\w$ have the form $\w_k$ where $(k, m) = 1$.  There are $\phi(m)$ of these, which is even for all $m \ge 3$.  If $\w_k$ is a conjugate, then $\w_{m - k}$ is also a conjugate, since $(k, m) = 1$ implies there exist integers $a, b$ such that $ak + bm = 1$, so $-a(m - k) + (b + a)m = 1$, and so $(m - k, m) = 1$.

For each conjugate $\w_k$, $\w_k \ne \w_{m - k}$; if this were the case, $k = -k \ (m)$, so $2k = 0 \ (m)$ and so $k$ would divide $m$, contradicting $(k, m) = 1$.  Therefore all the conjugates are distinct.

Finally, for each conjugate $\w_{k}$, $\w_k \cdot \w_{m - k} = 1$, so in computing the norm of $\w$, all the conjugates cancel out and the norm of $\w$ is seen to be $1$.

\item[34. (a)] Show that $1 + \w + \w^2 + \ldots + \w^{k - 1}$ is a unit in $\Z[\w]$ if $k$ is relatively prime to $\w$.

\[ (1 + \w + w^2 + \ldots + \w^{k-1}) \left(\frac{1 - w}{1 - \w^k}\right) = \frac{1 - \w^k}{1 - \w^k} = 1 \]
Therefore, if $\frac{1 - w}{1 - \w^k} \in \Z[\w]$ then $1 + \w + \ldots + \w^{k-1}$ is a unit.  Since $(k, m) = 1$, then there exist $a, b \in \Z$ such that $ak + bm = 1$, and so $\w = \w^{ak + bm} = \w^{ak} \w^{bm} = \w^{ak}$.  Since $\w^{ak} = \w^{(m - a)k}$ for negative $a$, $a$ can be assumed to be positive.  We then have \[ \frac{1 - \w}{1 - \w^k} = \frac{1 - \w^{ak}}{1 - \w^{k}} = 1 + \w^{k} + \w^{2k} + \ldots + \w^{(a - 1)k} \in \Z[\w] \]
This implies $1 + \w + \w^2 + \ldots + \w^{k-1}$ is a unit in $\Z[\w]$.

\item[34. (b)] The conjugates of $1 - \w$ are $\w^{kp^{r-1}} - 1$ for $1 \le k \le p - 1$.  By (a), $1 - w^{k} = \frac{1 - \w}{1 + \w + \ldots + \w^{k}}$, so \[ \norm(1 - w) = (1 - \w)^{n}\left(\prod_{(j, p^r) = 1} \sum_{i = 0}^{j} \w^{i}\right)^{-1} \]
By (a) the sum of the $\w^{i}$ factors is a unit in $\Z[\w]$, so the inverse of the product of each of these is also a unit, call it $u$.  Therefore \[ \norm(1-w) = u(1 - w)^n\]  However as $f(x) = 1 + x^{p^{r-1}} + \ldots + x^{(p-1)p^{r-1}}$ is the $p^r$th cyclotomic polynomial, the norm of $1 - w$ is the constant coefficient of the polynomial $1 + (1 - x)^{p^{r-1}} + \ldots + (1 - x)^{(p - 1)p^{r-1}} = p$, and so $\norm(1 - w) = p$.  Setting both sides equal to one another gives $p = u(1 - \w)^n$.

\item[35. (a)]  Let $\w = e^{2\pi i /m}$ and $\theta = \w + \w^{-1}$.  Then $\w^2 - (\w + \w^{-1})(\w) + 1 = 0$ and so $\w$ is a root of the polynomial $x^2 + \theta x + 1$.  $\w \not\in \Q[\theta]$, therefore $\Q[\w] : \Q[\theta]$ has degree 2.

\item[35. (b)]  Since $\theta = \w + \w^{-1} \in \R$, clearly $\Q[\theta]  \subseteq \Q[\w] \cap \R$.  We therefore have the tower of field extensions $\Q[\theta] \subseteq \Q[\w] \cap \R \subsetneq \Q[\w]$.  By (a), $[\Q[w] : \Q[\theta]] = 2$.  By the Tower Law, $[\R \cap \Q[\w] : \Q[\theta]]$ must be a divisor of 2 by distinct from 2 (since $w \not\in \R)$.  Therefore the degree must be 1 and so $\R \cap \Q[\w] = \Q[\theta]$.

\item[35. (c)]  Let $\sigma$ be the automorphism defined by $\sigma(\w) = \w^{-1}$.  Then $\sigma(\theta) = \theta$, and so $\Q[\theta]$ is in the fixed field of the automorphism $\sigma$.  As the degree of $\Q[\w]$ over $\Q[\theta]$ is 2, there can be no distinct intermediate field between $\Q[\w]$ and $\Q[\theta]$.  $\Q[\w]$ is not in the fixed field of $\sigma$ and so $\Q[\theta]$ must be the fixed field of this automorphism.

\item[35. (d)]  Show that $\mathbb{A} \cap \Q[\theta] = \R \cap \Z[\theta]$.
\begin{align*}
    \mathbb{A} \cap \Q[\theta]
        &= \mathbb{A} \cap (\R \cap \Q[\w]) & \text{By 35 (b).}\\
        &= (\mathbb{A} \cap \Q[\w]) \cap \R & \text{By associativity of intersection}\\
        &= \Z[\w] \cap \R & \text{By Theorem 12, Corollary 2}
\end{align*}

This is the required result.

\item[35. (e)] Let $n = \phi(m) / 2$.  The set $\{ 1, \w, \w^2, \ldots, \w^{n-1}, \w^{n}, \w^{n+1}, \ldots, \w^{m-1} \}$ is an integral basis for $\Z[\w]$.

Since $w^{n-k} = \w^{-k}$, we can write this basis as $\{1, \w, \w^{-1}, \w^2, \w^{-2}, \ldots, \w^{-n}\}$ instead (note $\w^{n} = \w^{-n}$).  We examine the set $\{1, \w, \theta, \theta\w, \theta^2, \theta^2\w, \ldots, \theta^{n} \}$.

Now we pair up the expressions $\theta^{k}\w$ with $\w^{k+1}$ and $\theta^{k}$ with $\w^{-k}$:

\begin{gather}
\{1, \w, \w^{-1}, \w^2,     \w^{-2},  \w^{3},     \ldots, \w^{n} \} \\
\{1, \w, \theta,  \theta\w, \theta^2, \theta^2\w, \ldots, \theta^{n-1}\w \}
\end{gather}

We evaluate the expression $\theta^{k}$ using the Binomial Theorem:

\[ \theta^{k} = (\w + \w^{-1})^k = \sum_{i = 0}^{k} \binom{k}{i}\ \w^{i} \w^{-(k-i)} = \sum_{i = 0}^{k} \binom{k}{i}\ \w^{2i - k} \]

Therefore

\[ \theta^{k}\w = \sum_{i = 0}^{k} \binom{k}{i}\ \w^{2i - k + 1} \]

For $\theta^{k}$, the power of $\w$ ranges between $-k$ and $k$ for $\theta^{k}$, and it uses 1 term of the power $\w^{-k}$ and no power of $\w$ with absolute value greater than $k$.

For $\theta^{k}\w$, the power of ranges between $-k + 1$ and $k + 1$ for $\theta^{k}\w$.  It uses 1 power of $\w^{k}$ and no other power of $\w$ with absolute value of greater than or equal to $k$.

Therefore, there is a lower triangular translation matrix $A$ between the basis (1) and (2).  $A$ has all 1s in the diagonal, and so $A$ has determinant 1 and is invertible over $\Z$.  Since (1) is an integral basis of $\Z[\w]$, so is (2).

\[
    A\ =\hspace{5mm}
    \bordermatrix{
             & 1 & \w& \w^{-1} & \w^2 & \w^{-2} & \ldots \cr
    1        & 1 &  0  & 0 & 0 & 0 & \ldots \cr
    \w       &  0  & 1 & 0 & 0 & 0 & \ldots \cr
    \theta   &  0  & 1 & 1 & 0 & 0 & \ldots \cr
    \theta\w &  1  & 0 & 0 & 1 & 0 & \ldots \cr
    \theta^2 &  2  & 0 & 0 & 1 & 1 & \ldots \cr
    \vdots   & \vdots & \vdots & \vdots & \vdots & \vdots & \ddots & \cr
    }
\]

\item[35. (f)] Show that $\{ 1, \theta, \theta^2, \ldots, \theta^{n-1} \}$ is an integral basis for $\mathbb{A} \cap \Q[\theta]$.

By (d), $\mathbb{A} \cap \Q[\theta] = \R \cap \Z[\theta]$, and by (e), any member $\alpha$ of $\Z[\theta]$ is expressible in terms of the basis vectors $\{ 1, \w, \theta, \theta\w, \theta^2, \ldots \}$:
\[ \beta = a_0 + a_1 \w + a_2 \theta + a_3 \theta \w + \ldots + a_{m-1} \theta^{n-1} \]

Since $\beta \in \mathbb{R}$, $\sigma(\beta) = \beta$ (where $\sigma$ is complex conjugation).  Therefore:
\begin{eqnarray*}
\beta &=& \sigma(a_0 + a_1 \w + a_2 \theta + a_3 \theta\omega + \ldots + a_{m-1}\theta^{n-1})\\
    &=& \sigma(a_0) + \sigma(a_1 \w) + \sigma(a_2 \theta) + \sigma(a_3 \theta\omega) + \ldots + \sigma(a_{m-1}\theta^{n-1}) \\
    &=& a_0 + a_1 \sigma(\w) + a_2 \sigma(\theta) + a_3 \theta\sigma(\omega) + \ldots + a_{m-1}\theta^{n-1} \\
    &=& a_0 + a_1 \w^{-1}+ a_2 \theta + a_3 \theta\sigma(\omega) + \ldots + a_{m-1}\theta^{n-1}
\end{eqnarray*}
Since the elements of basis are linearly independent, each odd $a_{i}$ must be equal to 0, and so $\beta$ must be expressible as $a_0 + a_2\theta + \ldots + a_{m-1}\theta^{n-1}$, and so $\Q[\theta]$ is an integral basis for $\mathbb{A} \cap \Q[\theta]$.

\item[35. (g)] Let $p$ be an odd prime.  Use exercise 23 to show that $\disc{\theta} = \pm p^{(p-3)/2}$.  Show the plus sign must hold.

By Exercise 23,
\begin{eqnarray*}
\disc{1, \w, \theta, \theta\w, \ldots, \theta^{n-1}} &=&
    \disc{\theta}^2 \norm^{\Q[\theta]}_{\Q} \text{disc}^{\Q[\w]}_{\Q[\theta]}(\w) \\
    p^{p - 2} &=& \disc{\theta}^2 \norm^{\Q[\theta]}_{\Q} (2\w - \theta) \\
    &=& \disc{\theta}^2 \norm^{\Q[\theta]}_{\Q} (\w - \w^{-1}) \\
    &=& \disc{\theta}^2 \norm^{\Q[\theta]}_{\Q} (\w^{-1}(\w + 1 )(\w - 1))\\
    &=& \disc{\theta}^2 p\\
    \pm p^{(p - 3)/2} &=& \disc{\theta}
\end{eqnarray*}
As pointed out in the exercise, the square root of the discriminant is present in $\Q[\theta]$.  Since $\Q[\theta] \subseteq \R$, $\disc{\theta} = p^{(p-3)/2}$.

\item[37.] Let $\alpha$ be an algebraic integer of degree $n$ over $\Q$ and let $f$ and $g$ be polynomials over $\Q$, each of degree $< n$, such that $f(\alpha) = g(\alpha)$.  Show $f = g$.

Let $h(x)$ be the minimal polynomial for $\alpha$.  If $f(\alpha) = g(\alpha)$, then $(f - g)(\alpha) = 0$.  Since $h$ is the minimum polynomial for $\alpha$, $h \mid f - g$.  However, $f - g$ has degree $< n$, and so $f - g = 0$.  Therefore $f = g$.

\item[40. (a)] Show $\disc{\alpha} = (d_1 d_2 \cdots d_{n-1})^2 \disc{R}$.

We first show $\disc{\alpha} = \disc{1, f_1(\alpha), \ldots, f_{n-1}(\alpha)}$.  \[ \disc{\alpha} = \disc{1, \alpha, \ldots, \alpha^{n-1}} \]  Since $f_{n-1}$ is a monic polynomial with degree $n-1$ it is a linear combination of $\alpha, \ldots, \alpha^{n-1}$, and so generate the same additive subgroup of $R_{k}$.  By Exercise 26, \[ \disc{1, \alpha, \ldots, \alpha^{n-1}} = \disc{1, \alpha, \ldots, \alpha^{n-2}, f_{n-1}(\alpha)} \]  Proceeding in this way we have \[ \disc{\alpha} = \disc{1, f_1(\alpha), \ldots, f_{n-1}(\alpha)} \]
Finally, we have
\begin{eqnarray*}
    \disc{R} &=& \disc{1, f_1(\alpha)/d_1, \ldots, f_{n-1}(\alpha)/d_{n-1}} \\
    &=& \frac{1}{d_1^2 \cdots d_{n-1}^2} \disc{1, f_1(\alpha)/d_1, \ldots, f_{n-1}(\alpha)/d_{n-1}} \\
    &=& \frac{1}{(d_1 \cdots d_{n-1})^2} \disc{\alpha} \\
\end{eqnarray*}
Multiplying both sides by $(d_1 \cdots d_{n-1})^2$ gives the required result.

\item[40. (b)] We show that $R_{k} / \Z[\alpha]$ has order $d_1, \ldots, d_k$ by induction on $k$.  Since $R = R_{n-1}$ the result with follow by induction.

For the base case we see that $1 / \Z[\alpha]$ has order 1.  Next let $R_{k} = R_{k-1} \oplus\frac{1}{d_k} f_k(\alpha) \Z$, so \[ R_{k} / \Z[\alpha] = R_{k-1}/\Z[\alpha] \oplus \frac{1}{d_k} f_k(\alpha)/\Z[\alpha] \]

By induction $R_{k-1}/\Z[\alpha]$ has order $d_1\cdots d_{k-1}$.  $f_{k}$ is a monic polynomial in $\alpha$ and so $f_k(\alpha) \in \Z[\alpha]$, therefore $\frac{1}{d_k} f_k(\alpha)/\Z[\alpha] = \frac{1}{d_k}$ which has order $d_k$, so the order of $R_k = d_1 \cdots d_{k}$.

\item[40. (c)] Show if $i + j < n$ then $d_i d_j \mid d_{i+j}$.

Since $f_i(\alpha) / d_i$ and $f_j(\alpha) / d_j$ are members of the ring $R$, $f_i(\alpha)f_j(\alpha) / d_i d_j$ must also be a member of the ring $R$.  $f_i(\alpha)f_j(\alpha)$ has order $i + j$.  Since this is $< n$, this element by be generated by the basis elements of order $\le i + j$.  Let $a_{k}$ be the integers that generate this element.  Then
\begin{eqnarray*}
    \frac{f_i(\alpha)f_j(\alpha)}{d_i d_j} &=& a_{i+j}\frac{f_{i+j}(\alpha)}{d_{i+j}} + \sum_{k = 0}^{i + j - 1} a_k \frac{f_{k}(\alpha)}{d_k} \\
    f_i(\alpha)f_j(\alpha) &=& a_{i+j}d_{i}d_{j}\frac{f_{i+j}(\alpha)}{d_{i+j}} + \text{Lower terms}
\end{eqnarray*}

We know $a_{i+j} \neq 0$.  Since $f_i$, $f_j$, and $f_{i+j}$ are each monic, the denominator must cancel with no remainder, giving $d_{i+j} = a_{i+j} d_i d_j$.  Therefore $d_i d_j \mid d_{i+j}$.

\item[40. (d)] Take $\frac{f_1(\alpha)}{d_1}$ as the basis element of order 1, and raise this element to the $i$-th power.  Each $(\frac{f_1(\alpha)}{d_1})^i$ is a polynomial of order $i$ and so generated by the basis element $\frac{f_i(\alpha)}{d_i}$.  By a similar argument as in 40. (c) (each of these terms is a monic polynomial and so the denominators must cancel with no remainder), $d_1^i \mid d_i$.

Let $j_i$ be the remainder left when dividing $d_i$ by $d_1^i$ ($j_1 = 1$).  Then:
\begin{eqnarray*}
    \disc{\alpha} &=& (d_1 \cdots d_{n-1})^2 \disc{R} \\
                  &=& (d_1 d_1^2 \cdots d_1^{n-1} \prod_{i=0}^{n-1} j_i)^2 \disc{R} \\
                  &=& (d_1^{n(n-1)/2})^2 (\prod_{i=0}^{n-1} j_i)^2 \disc{R}\\
                  &=& d_1^{n(n-1)} (\prod_{i=0}^{n-1} j_i)^2 \disc{R}
\end{eqnarray*}
Therefore $d^{n(n-1)} \mid \disc{\alpha}$.

\item[41. (a)] Let $m$ be a cubefree integer, $\alpha = \sqrt[3]{m}$, and write $m$ as $hk^2$ with $h, k$ relatively prime.  Let $R = \mathbb{A} \cap \Q[\alpha]$.  (Therefore $k^2$ has any square factors of $m$.).  Show $\disc{\alpha} = -27m^2$ (the 2018 edition has a typo).

Let $f(x) = x^3 - m$; $f(x)$ is the minimum polynomial of $\alpha$ over $\Q$ and has degree 3 (not $\equiv 0, 1\ (4)$), so $\disc{\alpha} = -\norm(f'(\alpha))$.  $f'(\alpha) = 3\alpha^2$ so $\alpha f'(\alpha) = 3m$ and $f'(\alpha) = 3m / \alpha$.  Note $\norm(\alpha) = m$ so $\norm(\alpha^{-1}) = 1/m$.  Therefore we have
\begin{gather*}
    \norm(3m/\alpha) = 27m^3 \norm(\alpha^{-1}) = 27m^2 \\
    \disc{\alpha} = -27m^2
\end{gather*}
Using Exercise 40, we see $-27m^2 = (d_1 d_2)^2 \disc{R}$ and $d_1^2 | d_2$, so writing $d_2 = d_1^2 j$, we have \[ -27m^2 = d_1^4 j^2 \disc{R} \]

Since $d_1$ has a sextic factor on the righthand-size, the only possibilities for $d_1$ are 1 or 3.  If $d_1 = 3$ then $9 \mid m$.

\item[41. (b)]  Show $d_1 = 1$ even when $9 \mid m$.

Suppose $9 \mid m$ and $d_1 = 3$.  Then $R$ has an integral basis with $1$ and $(\alpha + a)/3$ as two of the three basis vectors.

Let $\beta = (\alpha + a) / 3$ for some integer $a$.  As suggested in the exercise hint we consider the trace of $\beta^3$.  First, we determine the trace of $\alpha$ and $\alpha^2$ as these will be important to evaluate $\trace{\beta}$.
\begin{gather*}
    \trace{\alpha} = \alpha + \w\alpha + \w^2\alpha = \alpha(\w^2 + \w + 1) = 0\\
    \trace{\alpha^2} = \alpha^2 + \w^2 \alpha^2 + \w\alpha^2 = \alpha^2(\w^2 + \w + 1) = 0
\end{gather*}
With these in hand we now have \[ \beta^3 = \frac{(\alpha + a)^3}{27} = \frac{m + 3\alpha^2 a + 3a^2 \alpha + a^3}{27} \]

By the additive linearity of trace, we have
\begin{eqnarray*}
    \trace{\beta^3} &=& \frac{m}{9} + \frac{3a}{27}\trace{\alpha^2} + \frac{3a^2}{27}\trace{\alpha} + \frac{3a^3}{27}\\
    &=& \frac{m}{9} + \frac{3a^3}{27} \\
    &=& \text{Integer} + \frac{3a^3}{27}
\end{eqnarray*}

Since $\beta$ is an algebraic integer, $\beta^3$ is also an algebraic integer, and its trace must be a member of $\Z$.  Therefore $\frac{3a^3}{27}$ must be an integer, and so $27$ must divide $3a^3$, which means that $9$ divides $a^3$ and so 3 divides $a$.

Since 3 divides $a$, $\frac{\alpha + a}{3} = \frac{\alpha}{3} + \text{Integer}$.  Therefore $\alpha/3$ is a member of $R$, so $(\alpha / 3)^3 = m / 27 \in R$.  However, $m$ is cubefree and so $m / 27 \not \in \Z$.  This contradicts Corollary 1 of Theorem 1 - the only members of $\Q$ that are algebraic integers are members of $\Z$.

Therefore $d_1 = 1$ in all cases, and so $R$ has a basis containing $1$ and $\alpha$.  The third basis vector has yet to be determined.

\item[41. (c)] Write $m$ as $hk^2$.  Then $(\alpha^2 / k)^3 = m^2 / k^3 = (h^2 k^4)(k^3) = h^2 k$, and so $\alpha^2 / k$ is the root of the polynomial $f(x) = x^3 - h^2k$, and so $\alpha^2 / k \in R$.

\item[41. (d)] Suppose $m \equiv \pm 1\ (9)$.  Let $\beta = (\alpha \mp 1)^2 / 3$.  Show that
\[ \beta^3 - \beta^2 + \frac{1 \pm 2m}{3} \beta - \frac{(m \mp 1)^2}{27} = 0 \]

As suggested we calculate $(\beta - 1/3)^3$ in two ways:
\begin{eqnarray*}
(\beta - 1/3)^3 &=& ((\alpha \mp 1)^2 / 3 - 1/3)^3 \\
\beta^3 - \frac{3\beta^2}{3} + \frac{3\beta}{9} - \frac{1}{27} &=& \frac{(\alpha(\alpha \mp 2))^3}{27} \\
\beta^3 - \beta^2 + \frac{\beta}{3} - \frac{1}{27} &=& m\left(\frac{m \mp 6\alpha^2 + 12\alpha \mp 8}{27}\right)\\
\beta^3 - \beta^2 + \frac{\beta}{3} - \frac{m^2 \mp 2m + 1}{27} &=& m\left( \frac{\mp 6\alpha^2 + 12\alpha \mp 6}{27}\right)\\
\beta^3 - \beta^2 + \frac{\beta}{3} - \frac{(m \mp 1)^2}{27} &=& \mp \frac{2m}{3}\left( \frac{\alpha^2 \pm 2\alpha + 1}{3}\right) = \mp \frac{2m}{3}\beta
\end{eqnarray*}
Moving the terms around, we have the required result:
\[ \beta^3 - \beta^2 + \frac{1 \pm 2m}{3}\beta - \frac{(m \mp 1)^2}{27} = 0 \]

Since $m \equiv \pm 1\ (9)$, $1 \pm 2m$ is divisible by 3, and $m \mp 1$ is divisible by 9, so $(m \mp 1)^2$ is divisible by 27.  Therefore $\beta$ is the root of a monic polynomial with integer coefficients and so $\beta \in R$.

\item[41. (e)] Using (c) and (d), show that if $m \equiv \pm 1\ (9)$ then
\[ \frac{\alpha^2 \pm k^2 \alpha + k^2}{3k} \in R \]

Since $\alpha^2 / k \in R$, we can adding $k\alpha + k$ to the element to see that \[ \frac{\alpha^2 + k^2\alpha + k^2}{k} \in R \]

Next, observe that $\modequiv{k^2}{1}{3}$ - it cannot be 0 since $\modequiv{m}{\pm 1}{9}$.  Therefore $(k^2 - 1)/3$ and $(k^2 + 2)/3$ are integers.  Taking $(\alpha^2 \mp 2\alpha + 1)/3$, we add $(k^2 - 1)/3$ to see that \[ \frac{\alpha^2 \mp 2\alpha + k^2}{3} \in R \]

Next we have
\[ \frac{\alpha^2 \mp 2\alpha + k^2}{3} \pm \frac{\alpha(k^2 - 2)}{3} = \frac{\alpha^2 \pm k^2\alpha + k^2}{3} \in R \]

Since $3 \nmid k$ and 3 is a prime, there exist integers $a, b$ such that $3a + bk = 1$.  Therefore
\begin{eqnarray*}
    b\left(\frac{\alpha^2 \pm k^2\alpha + k^2}{3}\right) + a\left(\frac{\alpha^2 \pm k^2\alpha + k^2}{k}\right) &=& \frac{(kb + 3a)(\alpha^2 \pm k^2\alpha + k^2)}{3k} \\
    &=& \frac{\alpha^2 \pm k^2\alpha + k^2}{3k} \in R
\end{eqnarray*}

This is the required result.

\item[41. (f)]  We have $\disc{\alpha} = -27m^2$.  By Exercise 40(a), $d_2^2 \disc{R} = \disc{\alpha} = -27m^2 = -27h^2 k^4$.  We know $k \mid d_2$ so write $d_2 = jk$, thus $j^2 k^2 \disc{R} = -27h^2 k^4$ and so $j^2 \disc{R} = -27h^2 k^2 = -27mh$.  By assumption $h$ is squarefree, so $j^2 \mid -27m$, implying either $j \mid 3$ or $j \mid m$.  Therefore $j \mid 3m$.

\item[41. (g)]  Letting $p$ be a prime such that $p \neq 3$, $p \mid m$, $p^2 \mid m$.  Let $p \mid d_2$, and write $d_2 = pj$.  Therefore if $(\alpha^2 + a\alpha + b) / d_2 \in R$, then \[ j(\alpha^2 + a\alpha + b) / d_2 = (\alpha^2 + a\alpha + b) / p \in R \]  Since $(\alpha^2 + a\alpha + b)/p \in R$, its trace must be an integer; however $\trace{\alpha^2} =\trace{\alpha} = 0$, and so $3b / p \in \Z$.  $p \neq 3$, therefore $p \mid b$.  Therefore $(\alpha^2 + a\alpha) / p \in R$.

\[ \trace{((\alpha^2 + a\alpha)/p)^3} = \trace{(m^2 + a^3 m)/p^3} \]

Therefore $p^3 \mid 3(m^2 + a^3 m)$.  Since $p \neq 3$, $p^3 \mid m(m + a^3)$.  $m$ is cubefree and $p^2 \nmid m$, so $p^2 \mid m + a^3$.  Therefore $a^3 \equiv 0\ (p)$, meaning $a \equiv 0\ (p)$.  Considering the equation modulo $p^2$ we then have $m \equiv 0\ (p^2)$, a contradiction.  Therefore this case is impossible.

\item[41. (h)] Let $p \neq 3$ and $p^2 \mid m$.  By the previous problem $(\alpha^2 + a\alpha)/p^2 \in R$ and so we consider the trace:
\[ \trace{((\alpha^2 + a\alpha)/p^2)^3} = \trace{(m^2 + a^3 m)/p^6} \]
Therefore $p^6 \mid m(m + a^3)$.  Since $p^2 \mid m$, $p^4 \mid m + a^3$.  Considering the equation modulo $p^2$, we must have $a^3 \equiv 0\ (p^2)$, so $p^2 \mid a^3$.  Therefore $p \mid a$ and so $p^3 \mid a^3$.  Therefore $m + a^3 \equiv 0\ (p^3)$ and so $m \equiv 0\ (p^3)$ again contradicting $m$ cubefree.

Together with (g) this shows that $d_2$ has no common prime factor with $m$ that is not equal to 3.

\item[41. (i)] Take $(\alpha^2 + a\alpha + b)/d_2$.
\begin{eqnarray*}
    \frac{(\alpha^2 + a\alpha + b)^2}{d_2^2} &=& \frac{m\alpha + 2am + 2\alpha^2b + a^2\alpha^2 + 2ab\alpha + b^2}{d_2^2} \\
     &=& \frac{\alpha^2 (a^2 + 2b) + \alpha(m + 2ab) + (2am + b^2)}{d_2^2}
\end{eqnarray*}
Since this is an element of the ring and the basis element of order 2 has denominator $d_2$, $d_2$ must divide each of $a^2 + 2b$, $m + 2ab$, and $2am + b^2$.

\item[41. (j)]  We now consider what power of 3 divides $d_2$.  We know $d_2 \mid 3m$.  If $3 \nmid m$, then $9 \nmid d_2$.  Therefore, if $m \equiv \pm 1\ (9)$, $d_2 = 3k$; it cannot be any non-3 prime dividing $m$ by (g) and (h), and 9 does not divide $m$.

We now consider the case where $m \nequiv \pm 1\ (9)$ and $3 \nmid m$.  We assume $3 \mid d_2$ (to show a contradiction).

We evaluate the congruences obtained in (i) modulo 3.  Since $a^2 + 2b \equiv 0\ (3)$, $a^2 - b \equiv 0\ (3)$, and so $b \equiv a^2\ (3)$.  Substituting $b$ with $a^2$ in the equation $m + 2ab \equiv 0\ (3)$, we have $m + 2a^3 \equiv 0\ (3)$ and so $m - a^3 \equiv m - a \equiv 0\ (3)$, so therefore $a \equiv m\ (3)$.  Substituting $m$ for $a$ in the equivalence $b^2 + 2am \equiv 0\ (3)$, we have $b^2 \equiv -2a^2 \equiv a^2 \ (3)$.  Therefore since $a^2 + 2b \equiv 0\ (3)$, we have $b(b + 2) \equiv b(b - 1) \equiv 0\ (3)$.  $b \not\equiv 0\ (3)$ (as this would imply $m \equiv 0\ (3)$) so we must have $b \equiv 1\ (3)$.

Therefore we can write the basis element of order 2 as $\frac{\alpha^2 + (m + 3l)\alpha + (3j + 1)}{3i}$ for some $i, l, j$, and so by multiplying through by $i$ and subtracting the term $3l\alpha + 3j$ from the resulting fraction, we have:

\[ \frac{\alpha^2 + m\alpha + 1}{3} \in R \]

We now proceed by case on $m$ congruence to 3.  (Almost there!)

Suppose $\modequiv{m}{1}{3}$.  Then $\frac{\alpha^2 + \alpha + 1}{3} \in R$ and so by subtracing $\alpha$, $\frac{\alpha^2 - 2\alpha + 1}{3} = \frac{(\alpha - 1)^2}{3} \in R$.

We raise this to the fourth power and take the trace.  The only terms that contribute to the trace are those where $\alpha$ is raised to a power divisible by 3, so we have:
\begin{eqnarray*}
    \trace{\frac{(\alpha - 1)^8}{3^4}} &=& \frac{3}{3^4}\left(\binom{8}{6}\alpha^6 (-1)^2 + \binom{8}{3}\alpha^3 (-1)^5 + (-1)^8\right)\\
    &=& \frac{1}{27} \left(28m^2 - 56m + 1\right)
\end{eqnarray*}

Therefore, 27 must divide $28m^2 - 56m + 1$.  Congruent to 9, this equation reduces to $m^2 - 2m + 1 \equiv 0\ (9)$ so $(m - 1)^2 \equiv 0\ (9)$ and $\modequiv{m}{1}{9}$.  This contradicts $m \not\equiv \pm 1\ (9)$.  So $m$ cannot be congruent to 1 mod 3.

Next, suppose $m \equiv 2\ (3)$.  Threefore $\frac{\alpha^2 + 2\alpha + 1}{3} = \frac{(\alpha + 1)^2}{3} \in R$.  Again we raise this to the fourth power and take the trace.  (The equation is the same except for the negative terms.)

\[ \trace{\frac{(\alpha + 1)^8}{3^4}} = \frac{1}{27}\left(28m^2 + 56m + 1\right) \]
Modulo 9 we have $\modequiv{m^2 + 2m + 1}{0}{9}$ so $\modequiv{(m+1)^2}{0}{9}$ and so $\modequiv{m}{-1}{9}$, again contradicting $\modnotequiv{m}{\pm 1}{9}$.

Therefore if $3 \nmid m$ and $\modnotequiv{m}{\pm 1}{9}$, $3 \nmid d_2$.

\item[41. (k)]  Suppose $3 \mid m$ but $9 \nmid m$.  We assume $3 \mid d_2$ to show a contradiction.  By (i), $\modequiv{a^2 + 2b}{0}{3}$, so $\modequiv{a^2}{b}{3}$ (*).  Plugging this into $\modequiv{m + 2ab}{0}{3}$ we have $\modequiv{m - a^3}{0}{3}$.  Since $\modequiv{a^3}{a}{3}$, we thus have $\modequiv{m}{a}{3}$ and so $\modequiv{a}{0}{3}$, and also $\modequiv{b}{0}{3}$ by (*).

Therefore we can write the basis element of order 2 as $\frac{\alpha^2 + 3i\alpha + 3j}{3l}$, and by multiplying through by $l$ and subtracting $i\alpha + j$, we have $\frac{\alpha^2}{3} \in R$.  Cubing this element and taking the trace we must have $m^2  / 9 \in \Z$, contradicting $9 \nmid m$.  Therefore $3 \nmid d_2$.

\item[41. (l)]  Suppose $9 \mid m$.  We show $9 \nmid d_2$.  Assume $9 \mid d_2$ (to show a contradiction).  By (i), $9 \mid ab$ and $9 \mid b^2$ so either $9 \mid b$ or $3 \mid b$.  Assume $3 \mid b$, therefore since $\modequiv{a^2 + 2b}{0}{9}$, we must have $\modequiv{a^2}{-6 \equiv 3}{9}$.  However, 3 is not the square of any element mod 9, so this equation is unsatisfiable.  We must have $9 \mid b$.

Therefore, $(a^2 + a\alpha) / 9 \in R$.  Taking this to the third power and considering the trace, we must have $9^3 \mid 3(m^2 + ma^3)$ and $9^2 3 \mid m(m + a^3)$.  Since $m$ is cubefree and $9 \mid m$, therefore $27 \mid m + a^3$.  Considering $m + a^3$ modulo 9, we have $\modequiv{a^3}{0}{9}$; therefore $a$ must be congruent to 0, 3, or 6 modulo 9.  In all these cases we have $\modequiv{a^2}{0}{9}$.  Since $9^2 \mid a^3$ and $9^2 \mid (m + a^3)$, $9^2 \mid m$, which contradicts $m$ being cube-free.  Therefore $9 \nmid d_2$.

\item[43. (a)] Let $f(x) = x^5 + ax + b$ with $a, b \in \Z$ and $f$ irreducible over $\Q$.  Let $\alpha$ be a root of $f$.  Show $\disc{\alpha} = 4^4 a^5 + 5^5 b^4$.

We proceed in a similar fashion to Exercise 28: first, we determine $f'(\alpha)$, then we determine $\norm(f'(\alpha))$ by collecting the most and least significant the coefficients of its polynomial.

$f'(x) = 5x^4 + a$, so $\alpha f'(x) = 5\alpha^5 + a = -5(a\alpha + b) + a = -4a \alpha - 5b$ and $f'(\alpha) = (-4a\alpha - 5b) / \alpha$.  The expression $4a\alpha + 5b$ is a root of the polynomial $(\frac{x - 5b}{4a})^5 + a(\frac{x - 5b}{4a}) + b$.  The norm $\norm(4a\alpha + 5b)$ is the negative of the $x^0$ coefficient divided by the $x^5$ coefficient (again, negative because 5 is odd), so we calculate those values.

The $x^0$ coefficient is $(\frac{-5b}{4a})^5 + a(\frac{-5b}{4a}) + b = (\frac{-5b}{4a})^5 + \frac{-b}{4}$, and the $x^5$ coefficient is $(\frac{1}{4a})^5$, so $\norm(4a\alpha + 5b) = 5^5 b^5 + 4^4 a^5 b$.

Therefore, \[ \disc{\alpha} = \norm(-(4a\alpha + 5b)/\alpha) = -\frac{5^5 b^5 + 4^4 a^5 b}{-b} = 5^5 b^4 + 4^4 a^5 \]  This is the required result.  (The plus sign for the discriminant holds because $5 \equiv 1\ (4)$)

\item [43. (b)] Suppose $\alpha^5 = \alpha + 1$.  We are given that this polynomial is irreducible because it is irreducible modulo 3.  (The options are $0$, $1$, and $2$: $0^5 \not\equiv 0 + 1\ (3)$, $1^5 \not\equiv 1 + 1\ (3)$, and $2^5 = 2 \not\equiv 1 + 2 = 0\ (3)$.)

In this case $a = -1$ and $b = -1$ so the above formula gives $\disc{\alpha} = 5^5 - 4^4 = 125 \cdot 25 - 16 \cdot 16 = 2869 = 19 \cdot 151$.  Since the discriminant is squarefree, $\mathbb{A} \cap \Q[\alpha] = \Z[\alpha]$.

\item[43. (c)] Let $a$ be squarefree and not equal to $\pm 1$.  Let $\alpha$ be a root and $d_1, d_2, d_3, d_4$ be as in Theorem 13.  Prove that if $4^4 a + 5^5$ is squarefree then $d_1 = d_2 = 1$ and $d_3 d_4 \divides a^2$.

By exercise 40, \[ \disc{\alpha} = 5^5 a^4 + 4^4 a^5 = a^4(5^5 + 4^4 a) =(d_1 d_2 d_3 d_4)^2 \disc{R} \]

Here $d_1 d_2 \divides d_3$, $d_1 d_2 \divides d_4$, and $d_1 d_3 \divides d_4$.  Therefore $d_1$ and $d_2$ both have 6 factors represented in the $\disc{\alpha}$ expression which is impossible unless they are both $1$.  Since $5^5 + 4^4 a$ is squarefree, $(d_3 d_4)^2$ must divide $a^4$ and so $d_3 d_4 \divides a^2$.

Verify that $4^4 a + 5^5$ is squarefree when $a = -2, -3, -6, -7, -10, -11, -13$, and $-15$.

\begin{verbatim}
sage: [(factor(x), is_squarefree(x)) for x in
       map(lambda a: 5^5 + 4^4 *a,
       [-2, -3, -6, -7, -10, -11, -13, -15])]

[(3 * 13 * 67, True),
    (2357, True),
    (7 * 227, True),
    (31 * 43, True),
    (5 * 113, True),
    (3 * 103, True),
    (-1 * 7 * 29, True),
    (-1 * 5 * 11 * 13, True)]
    \end{verbatim}

    Experimenting a bit more with Sage, we can quickly test integers using the following code:
    \begin{verbatim}
sage: def test_poly_degree_5(a):
....:     return (is_squarefree(5^5 + 4^4 *a) and
....:             is_squarefree(a))
....:
sage: filter(lambda x: test_poly_degree_5(x),
....:        range(2, 30))
[2, 3, 5, 6, 7, 10, 11, 14, 15, 17, 19, 21, 23, 26, 29]
sage: filter(lambda x: test_poly_degree_5(x),
....:        range(-2, -30, -1))
[-2, -3, -6, -7, -10, -11, -13, -15, -17, -19, -21,
 -22, -26, -29]
\end{verbatim}

\item[43. (d)] Let $\alpha$ be as in part (c) ($\alpha$ is the root of a polynomial $f(x) = x^5 + ax + a$).  Show $\alpha + 1$ is a unit.

We have $\alpha^5 = -a(\alpha + 1)$, so we take the norm of both sides.  $\norm(\alpha^5) = -a^5 = \norm(-a) \norm(\alpha + 1) = -a^5 \norm(\alpha+1$, so $\norm(\alpha + 1) = 1$.  Therefore $\alpha + 1$ is a unit in $\mathbb{A} \cap \Q[\alpha]$.

\item[44. (a)] Let $f(x) = x^5 + ax^4 + b$ where $a, b \in \Z$, and let $\alpha$ be a root of $f$.  To determine the discriminant of $\alpha$, we proceed as in exercise 28 and 43.  The derivative of $f(x)$ is $f'(x) = 5x^4 + 4ax^3$, so  \[ f'(\alpha) = \alpha^3(5\alpha + 4a) \]  $\norm(a^3) = -b^3$ so determine the norm of $5\alpha + 4a$ by observing it is the root of the polynomial $(\frac{x - 4a}{5})^5 + (\frac{x - 4a}{5})^4 + b$.  The $x^0$ term is $(\frac{-4a}{5})^5 + (\frac{-4a}{5})^4 + b$ while the $x^5$ term is $\frac{1}{5^5}$, \[ \norm(5\alpha + 4a) = (4a)^5 - 5a (4a)^4 - 5^5 b = -(4a)^5 \cdot (-4 + 5) - 5^5 b = -(4^5 a^5 + 5^5 b) \]  Therefore $\disc{\alpha} = (4^5 a^5 + 5^5 b)b^3$ as required (the discriminant is positive since $5 \equiv 1\ (4)$).

\item[44. (b)] TODO

\item[45.] Let $\alpha$ be the root of the polynomial $f(x) = x^n + ax + b$.  Find a formula for $\disc{\alpha}$.

We proceed in similar fashion to exercise 43. $f'(\alpha) = n\alpha^{n-1} + a$, so we have:
\begin{eqnarray*}
    \alpha f'(\alpha) &=& n\alpha + a\alpha \\
                      &=& -n(a \alpha + b) + a\alpha \\
                      &=& -((n - 1)a \alpha + bn) \\
    f'(\alpha) &=& -((n - 1)a \alpha + bn) / \alpha
\end{eqnarray*}

We now calculate $\norm((n - 1)a \alpha + bn))$.  This is the root of the polynomial
\[ g(x) = \left(\frac{x - bn}{(n-1)a}\right)^n + a \left(\frac{x - bn}{(n-1)a}\right) + b\]

The norm is equal to $(-1)^n$ times the $x_0$ coordinate multiplied by the inverse of $x_n$ coordinate.  Therefore,

\[ \norm((n-1)a \alpha + bn) = (bn)^n + (-1)^{n + 1} a^n b (n -1)^{n-1} \]

The inverse of the $x_n$ coordinate is seen to be $((n - 1)a)^n$

The discriminant is then (with the $\pm$ positive if $n \equiv 0, 1\ (4)$, negative otherwise):
\begin{eqnarray*}
    \disc{\alpha} &=& \frac{\pm (-1)^n \norm((n - 1)a \alpha + bn)}{b(-1)^n}\\
    &=& \frac{\pm (bn)^n + (-1)^{n + 1} a^n b (n -1)^{n-1}}{b} \\
    &=& \pm [b^{n-1} n^n + (-1)^{n + 1} a^n (n -1)^{n-1}]
\end{eqnarray*}

Plugging values in gives:
\begin{eqnarray*}
    n = 2 &=& -(2^2 b -  a^2) = a^2 - 4b \\
    n = 3 &=& -(27b^2) + a^3 2^2) = -27b^2 + 4a^3 \\
    n = 4 &=& b^3 4^4 - a^4 3^3 = 256b^3 - 27a^4 \\
    n = 5 &=& b^4 5^5 + a^5 4^4
\end{eqnarray*}
These agree with the known values of these polynomials.

Next, we calculate $\disc{\alpha}$ if $\alpha$ is a root of $x^n + ax^{n-1} + b$.  The derivative $f'(\alpha) = n\alpha^{n-1} + a(n-1)\alpha^{n-2} = \alpha^{n-2}(\alpha n + a(n-1))$, so \[ \disc{\alpha} = \pm \norm(f'(\alpha)) = \pm \norm(\alpha^{n-2})\norm(n\alpha + (n-1)a) \]

The norm $\norm(\alpha^{n-2}) = \norm(\alpha)^{n-2} = (-1)^{n} b^{n-2}$, so we only need to calculate $\norm(n\alpha + (n-1)a)$.  This is a root of the polynomial \[ \left(\frac{x - (n-1)a}{n}\right)^n + a\left(\frac{x - (n-1)a}{n}\right)^{n-1} + b \]

We now calculate the norm of this.  The $x_n$ coefficient is $\frac{1}{n^n}$, and the $x_0$ coefficient is \[ \left(-\frac{(n-1)a}{n}\right)^n + a\left(-\frac{(n-1)a}{n}\right)^{n-1} + b \]

Multiplying through by $n^n$ gives us:
\begin{eqnarray*}
    \norm(n\alpha + (n-1)a) &=& (-1)^n[(-1)^n (n-1)^n a^n + (-1)^{n-1} a^n (n-1)^{n-1} n + b n^n] \\
    &=& (n-1)^n a^n - a^n (n-1)^{n-1} n + (-1)^n b n^n \\
    &=& a^n (n - 1)^{n-1}(n-1 - n) + (-1)^n bn^n \\
    &=& -a^n (n - 1)^{n-1} + (-1)^n bn^n
\end{eqnarray*}
Multiplying the norm by $(-1)^n b^{n-2}$ we have
\[ \disc{\alpha} = \pm [ bn^n + (-1)^{n-1} a^n (n - 1)^{n-1}] b^{n-2} \]
This agrees with the answer to Exercise 44 (a) ($n = 5$) and I confirmed via Sage that the formula holds for some examples where $n = 4$ and $n = 6$:
\begin{verbatim}
sage: a = 4; b = -7; n = 4
sage: K.<g> = QQ.extension(x^4 + a*x^3 + b)
sage: K.disc([1, g, g^2, g^3])
-426496
sage: (b*n^n - a^n * (n - 1)^(n - 1))*b^(n-2)
-426496
sage: a = 3; b = -5; n = 6
sage: K.<g> = QQ.extension(x^6 + a*x^5 + b)
sage: K.disc([1, g, g^2, g^3, g^4, g^5])
1569628125
sage: -(b*n^n - a^n * (n - 1)^(n - 1))*b^(n-2)
1569628125
\end{verbatim}

\end{enumerate}

\section*{Chapter 3}

\begin{enumerate}
\item[2.] Prove that every finite integral domain $D$ is a field.

For $\alpha \in D$, consider the set $\{1, \alpha, \alpha^2, \ldots\}$.  Since $D$ is finite this set must also be finite, so there must be some $i, j$, $i \neq j$ such that $\alpha^i = \alpha^j$.  Thus $\alpha^{j - i} = 1$, and $\alpha^{j - i - 1}\alpha = \alpha^{j - i} = 1$, so every element in $D$ has an inverse, and $D$ is therefore a field.

\item[3.] Let $G$ be a free abelian group of rank $n$, with additive notation.  Show for any $m \in \Z$, $G / mG$ is the direct sum of $n$ cyclic group of order $m$.

Since $G$ is a free abelian group of rank $n$, \[ G \simeq \underbrace{\Z \oplus \cdots \oplus \Z}_\text{$n$ copies} \]
Therefore \[ G/mG \simeq \underbrace{\Z/m\Z \oplus \cdots \oplus \Z/m\Z}_\text{$n$ copies} \]
Each $\Z/m\Z$ is a cyclic group of order $m$, so the order of $G/mG$ is $m^n$.

\item[4.] Let $K$ be any number field of degree $n$ over $\Q$.  Prove that every nonzero ideal $I$ in $R = \ringofintegers{K}$ is a free abelian group of rank $n$.

As an additive subgroup of $R$, $I$ must be a free abelian group of order $\le n$.  Let $\{\beta_1, \ldots, \beta_n\}$ be a basis for $R$, and take $\alpha \in I$.  $\{\alpha\beta_1, \ldots, \alpha\beta_n\} \subseteq I \subseteq R$ is a free abelian group of order $n$.  Since $I$ contains $\alpha I$, the rank of $I$ must also be $n$.

\item[7.] If $I + J = 1$ then there exist $\alpha \in I, \beta \in J$ such that $\alpha + \beta = 1$.  Raising both powers to the $m + n$th power, we have $(\alpha + \beta)^{m + n} = 1^{m + n} = 1$.  By the binomial theorem, $(\alpha + \beta)^{m + n} = \sum_{k = 0}^{m + n} \binom{m + n - k}{k}\alpha^{m + n - k}\beta^{k}$.  If $k < n$, this element is a member of $I^{m}$ (as $\alpha^{n + \text{positive}} \in I^{m}$); otherwise this element is a member of $J^{n}$.  Therefore $(\alpha + \beta)^{m + n} \in I^{m} + J^{n}$.

\item[8. (a)] Suppose $I = (2, x)$ was generated by some $\alpha \in I$.  Therefore there are $\beta, \gamma \in \Z[x]$ such that $\alpha\beta = 2$ and $\alpha\gamma = x$.  Since $\alpha\beta = 2$, the rank of $\alpha$ must be 0; $\alpha \in \Z$.  The only option is $\alpha = 2$ (since $1 \not\in I$).  However 2 is not a factor of $x$ in $\Z[x]$.  Therefore the ideal $(2, x)$ is not principal in $\Z[x]$.

\item[8. (b)] Let $f, g \in \Z[x]$ and let $m, n$ be the gcd of the coefficients of $f$ and $g$ respectively.  Prove $mn$ is the gcd of the coefficients of $fg$.

Since $m$ and $n$ are the gcds of $f$ and $g$ we can write \begin{gather}f = m(a_0 + a_1 x + \ldots + a_j x^j) \\ g = n(b_0 + b_1x + \ldots + b_k x^k)\end{gather} where $(a_0, \ldots, a_j) = 1$ and $(b_0, \ldots, b_k) = 1$.  Let $d$ be the GCD of the coefficients of $fg$.  As \[ fg = mn(\sum_{0 \le l \le j} \sum_{0 \le m \le k} a_l b_m ) \] we know that $mn \mid d$.

Suppose there is some prime $p$ such that $p$ divides $a_l b_m$ for all $l, m$.  Since $(a_0, \ldots, a_j) = 1$ and $(b_0, \ldots, b_m) = 1$, there is some first $a_l$ and first $b_m$ such that $p \nmid a_l$ and $p \nmid b_m$; so $p \mid a_0, \ldots, a_{l-1}$ but $p \nmid a_{l}$ and similarly $p \mid b_0, \ldots, b_{m-1}$ but $p \nmid b_{m}$.  The $x^{l + m}$ term in $fg$ has coefficient $a_l b_m + a_{l + 1} b_{m-1} + \ldots a_{l-1} b_{m+1} + \ldots$.  Taken modulo $p$, $\modnotequiv{a_l b_m}{0}{p}$ but $p$ divides every other term in the expansion.  This contradicts $p$ being dividing the sum, and so there must be no other factor $d$ beyond $mn$.

\item[8. (c)] Let $f \in \Z[x]$ be irreducible over $\Z$.  Show $f$ is irreducible over $\Q$.

Suppose $f$ is irreducible over $\Z$ but reducible over $\Q$, i.e. $f = gh$ for $g, h \in \Q[x]$.  Then we can pull out the denominators from $g, h$, giving us $gh = \frac{g'h'}{d}$ where $g', h' \in \Z[x]$.  Let $a$ and $b$ tbe the GCD of the coefficients of $g'$ and $h'$ respectively.  We must have $ab != d$ because otherwise then $f$ would be reducible into the product of two polynomials in $\Z[x]$.  Therefore, reducing to lowest terms if necessary, we have $ab \nmid d$.  However, multiplying both sides of the equation by $d$ gives $df = g'h' = ab(g'' h'')$ for some $g''$ and $h''$ and so by (b), $ab \mid d$; this is a contradiction.  Therefore $f$ must be also irreducible over $\Q[x]$.

\item[9.] Let $K$ and $L$ be number fields, $K \subset L$, $R = \ringofintegers{K}, S = \ringofintegers{L}$.

\item[9. (a) - TODO] Let $I$ and $J$ be ideals in $R$ and suppose $IS \mid JS$.  Show $I \mid J$.

% Let $I = \p_1 \cdots \p_k$ and $J = \q_1 \cdots \q_m$ be the factorization of $I$ and $J$ into prime ideals in $R$.  Since $IS \mid JS$, there is an ideal $A \subset S$ such that $IAS = JS$.  Taking the prime factorization of $I$ and $J$ in $S$, we have $\p_1 \cdots \p_k A S = (\q_1 S) \cdots (\q_m S)$.

% Since factorization of ideals is unique, $\p_1$ must appear in the right-hand factorization, and so there must be some $\q_i S$ such that $\p_1 \mid \q_i S$; assume without loss of generality it is $\q_1$.

\item[10.] Show $e$ and $f$ are multiplicative in terms of towers.

Let $K \subset L \subset M$ and $R \subset S \subset T$ be the associated number fields and $P \subset Q \subset U$ prime ideals.

{\bf f is multiplicative:} By the third isomorphism theorem, there is the field series of field inclusions: $R / P \to S / Q \to T / U$.  $[S / Q : R / P] = f(Q|P)$ and $[T/U : S/Q] = f(U|Q)$; therefore the composition map from $R / P \to T / U$  must have degree $f(U|P) = f(Q|P)f(U|Q)$ by the tower law for field extensions.

{\bf e is multiplicative:} $P = Q^{e(Q|P)} I$ and $Q = U^{e(U|Q)} J$ for ideals $I, J$ such that $I + Q$ and $J + U$ are relatively prime.  Therefore $P = U^{e(U|Q)e(Q|P)} IJ$ with $U^{e(Q|P)e(U|Q)}$ and $IJ$ relatively prime; the factor of $U$ dividing $P$ is $e(U|P)$ so $e(U|P) = e(U|Q)e(Q|P)$.

\item[11.] Since $\alpha \in I$, $I \mid (\alpha)$, and so $I \cdot J = (\alpha)$.  Taking norms of both sides, $||I|| \cdot ||J|| = || (a) ||$.  By Theorem 22 (c), $|| (\alpha) || = \norm^{K}_{\Q}(\alpha)$, so $||I|| \mid \norm^{K}_{\Q}(\alpha)$, with equality holding if $I$ is principal.

\item[12. (a)] Verify that $5S = (5, \alpha + 2)(5, \alpha^2 + 3\alpha - 1)$ in $S = \Z[\sqrt[3]{2}]$, $\alpha = \sqrt[3]{2}$.

Let $I = (5, \alpha + 2)(5, \alpha^2 + 3\alpha - 1)$.  The generators of $I$ are: \setcounter{equation}{0}\begin{gather}5^2 \\ 5(\alpha^2 + 3\alpha -1)\\ 5(\alpha + 2)\\ (\alpha + 2)(\alpha^2 + 3\alpha -1) = \alpha^3 + (3 + 2)\alpha^2 + (-1 + 6)\alpha - 2 = 5(\alpha^2 + \alpha) \end{gather}
All generators have a factor of 5 so $1 \not\in I$; therefore $5 \subset I$.  We have $\alpha \cdot (3) - (1) + 3 \cdot (2) = 45$, so $\gcd(45, 5^2) = 5 \in I$.  Therefore $(3) - 10 = 5\alpha \in I$ and also $5\alpha^2 \in I$ by subtracting factors from (2); therefore $I = 5S$.

\item[12. (b)] Show there is an isomorphism between $\Z[x] / (5, x^2 + 3x - 1)$ and $\Z_{5}[x]/(x^2 + 3x - 1)$.

Let $a \in \Z[x] / (5, x^2 + 3x - 1)$.  Then $a$ can be associated with a coset representative $f(x) + 5g(x) + (x^2 + 3x - 1)h(x)$ where all of the coefficients of $f(x)$ and $h(x)$ are less than 5 (other terms can be placed in $g(x)$).

Let $\rho$ be the mapping of $\Z[x] \to \Z_5[x]$ by reducing the coefficients mod 5. $\rho(a) = \rho(f(x)) + (x^2 + 3x - 1)\rho(h(x)) = f(x) + (x^2 + 3x - 1)h(x)$ and so $\rho$ is an isomorphism from the quotient ring $\Z[x] / (5, x^2 + 3x - 1)$ to $\Z_5[x]/(x^2 + 3x - 1)$.

\item[12. (c)]  Show there is a surjective homomorphism from $\Z[x] / (5, x^2 + 3x - 1)$ onto $S / (5, \alpha^2 + 3\alpha - 1)$.

The ring homomorphism $\psi$ from $\Z[x] \to S$ defined by $\psi(x) = \alpha$ is a surjective.  Let $\beta \in S$; $\beta = m_0 + m_1\alpha + m_2\alpha^2$ for integers $m_0, m_1, m_2$, so $f(m_0 + m_1x + m_2x^2) = \beta$.  Therefore the surjective $\psi$ induces a surjection $\hat{\psi}$ on the quotient rings: \[ \Z[x] / (5, x^2 + 3x + 1) \to S / (5, \alpha^2 + 3\alpha - 1) \]

This utilizes the following lemma on ring homomorphisms:

\begin{lemma}
Let $R$ and $R'$ be rings and $\psi$ be a sujection $R \to R'$.  Let $I$ be an ideal of $R$.  Then the mapping that $\psi$ induces between the quotient groups $R / I \to R / \psi(I)$ is also a surjection.
\end{lemma}

\begin{proof}
    Take $a \in R/\psi(I)$; then $a = r' + \psi(I)$ for $r' \in R'$.  Since $\psi$ is surjective there must be some $r \in R$ such that $\psi(r) = r'$; therefore the coset $r + I$ is mapped to $r' + \psi(I)$, and the mapping between the quotient groups is also surjective.
\end{proof}

\item[12. (d)] In $\Z_5$, the polynomial $f(x) = x^2 + 3x - 1 = x^2 + 3x + 4$ is irreducible.  Any factor must be a a root, and manual testing gives $f(0) = 4, f(1) = 3, f(2) = 4, f(3) = 2$, and $f(4) = 2$, so the polynomial has no root and is irreducible.  Therefore $\Z_5 / (x^2 + 3x - 1)$ is a field of order $5^2 = 25$.

Let $I = (5, \alpha^2 + 3\alpha - 1)$.  By (b) and (c) there is a surjection $\hat{\psi}$ from $\Z_5 / (x^2 + 3x - 1)$ onto $S / I$.  As $\hat{\psi}$ is onto and the source ring has cardinality 25, $S / I$ must have a cardinality dividing 25; the options are 1 ($R = S$), 5, and 25 ($S/I \simeq \Z_5/(x^2 + 3x - 1)$).

Assume $|S / I| = 5$; we derive a contradiction.  Since $\alpha^3 = 2$, we must have $2 \not \in \kernel(\psi)$ (otherwise $\alpha^3 = 0$ and so $\alpha \in \kernel(\psi)$, giving $S \subset \kernel(\psi)$).  The only cube root of 2 modulo 5 is 3, so $\psi(\alpha) = 3$.  However then $\psi(\alpha^2) = 4 + I$, $\psi(3\alpha) = 4 + I$, and $\psi(-1) = 4 + I$; thus $\psi(\alpha^2 + 3\alpha - 1) = 2$.  But we know $\psi(\alpha^2 + 3\alpha - 1) = 0$.  This is a contradiction, so $|S/I| \neq 5$.

Therefore $I = (5, \alpha^2 + 3\alpha - 1)$ is either the whole ring or a prime ideal inducing $S / I$ to be a field of order 25.

\item[12. (e)] Suppose $(5, \alpha^2 + 3\alpha - 1) = S$.  Then by (a), $5S = (5, \alpha + 2)S$; however, $\alpha + 2 \not\in 5$, so $S/(5, \alpha^2 + 3\alpha - 1)$ must be a field of order 25.

\item[13. (a)] Let $S = \Z[\alpha]$, $\alpha^3 = \alpha + 1$.  Verify $23S = (23, \alpha - 10)^2 (23, \alpha - 3)$.

Let $I = (23, \alpha - 10)^2 (23, \alpha - 3)$.  The generators of $I$ are:\setcounter{equation}{0}\begin{gather}23^3 \\ 23^2 (\alpha -3) \\ 23^2 (\alpha - 10)\\ (\alpha - 10)^2 (\alpha - 3) = -23 (\alpha^2 - 7\alpha + 13) \\ 23 (\alpha - 10)^2 = 23(\alpha^2 - 20\alpha + 100) \\ 23(\alpha - 10)(\alpha - 3) = 23(\alpha^2 - 13\alpha + 30) \end{gather}
From the generators it is clear that 23 divides every member of $I$, and so $23S \subset I$. To show the required result we need to show $\{23, 23\alpha, 23 \alpha^2\} \in I$.
\begin{gather} (4) + (5) = 23(-13\alpha + 87) \\
    2 \cdot (6) - (5) + (4) = 23\alpha + 53 \cdot 23\\
    13 \cdot (8) - (7) = 23 \cdot 602
\end{gather}
From (1), (2), and (3), we must have $23^2 \in I$ as this is the GCD of (1) with the sum of (2) and (3); since $23 \cdot 602 \in I$, therefore $23 \in $I as it is the GCD of these two integers.  Subtracting a multiple of $23 \in $I from (8) gives us $23\alpha \in I$, and we thus have $23\alpha^2 \in I$ as well by subtracting the appropriate terms from (5) or (6).  This verifies $\{23, 23\alpha, 23\alpha^2\} \in I$ and so $23S = (23, \alpha - 10)^2 (23, \alpha - 3)$.

\item [13. (b)] Show that $(23, \alpha - 10, \alpha - 3) = S$.  Conclude that $(23, \alpha - 10)$ and $(23, \alpha - 3)$ are relatively prime.

Since $-10 \cdot [(\alpha - 10) - (\alpha - 3)] - 3 \cdot 23 = 1$, $(23, \alpha - 10, \alpha - 3) = S$.  Since $(23, \alpha -10) \mid 23S$ and $(23, \alpha -3) \mid 23S$, neither is the whole ring $S$ and so they must be relatively prime ideals in $S$.

\item[14.] Let $K$ and $L$ are number fields, $K \subset L$, $R = \ringofintegers{K}, S = \ringofintegers{L}$.  Assume $L$ is normal over $K$ and let $G$ be the Galois group of $L$ over $K$.  Let $|G| = [K : L] = n$.

\item[14. (a)] Suppose $Q$ and $Q'$ are two primes of $S$ lying over a prime $P$ of $R$.  Show the number of automorphisms $\sigma$ such that $\sigma(Q) = Q$ is the same number of $\sigma \in G$ such that $\sigma(Q) = Q'$.  Conclude this number is $e(Q|P)f(Q|P)$.

Enumerate the distinct automorphisms fixing $Q$ as $\sigma_0, \ldots, \sigma_k$, and the automorphisms taking $Q$ to $Q'$ as $\tau_0, \ldots, \tau_l$.  Let $\tau$ be one of the automorphisms taking $Q$ to $Q'$ (by Theorem 23, this must exist) and consider the automorphisms $\sigma_0\tau, \ldots, \sigma_k\tau$.  These are $k$ distinct automorphisms taking $Q$ to $Q'$ (if $\sigma_i \tau = \sigma_j \tau$ then $\sigma_i = \sigma_j$), so $k \le l$.  Conversely, consider the automorphisms $\tau \tau_0^{-1}, \ldots, \tau \tau_l^{-1}$ taking $Q$ to $Q$.  Each of these must be one of the $\sigma_{k}$, and each must be distinct; if $\tau \tau_i^{-1} = \tau \tau_j^{-1}$ then $\tau_i = \tau_j$, so $l \le k$.  Therefore $l = k$.

We count the number of permutations in $G$ so as to determine the number of permutations fixing $Q$ (call this number $k$).  For each prime $P$, there are $r$ distinct primes $Q_1, \ldots, Q_r$ lying over $P$, and so there are $k$ automorphisms taking $Q_1$ to $Q_1$, $k$ automorphisms taking $Q_1$ to $Q_2$, etc.  Therefore $n = kr$; since $n = re(Q|P)f(Q|P)$, $k = e(Q|P)f(Q|P)$.

\item[14. (b)] For an ideal $I \subset S$, define $\norm^{L}_{K}(I)$ to be the ideal $R \cap \prod_{\sigma \in G} \sigma(I)$.  Show that for a prime $Q$ lying over $P$, $\norm^{L}_{K}(Q) = P^{f(Q|P)}$.

Let $e = e(Q|P)$, $f = f(Q|P)$. and $Q_1, \ldots, Q_r$ be the ideals of $S$ lying over $P$.  By (a) there are $ef$ automorphisms sending $Q$ to $Q_1$, $Q$ to $Q_2$, etc.  Therefore
\begin{eqnarray*} \norm^{L}_{K}(I) &=& R \cap (Q_1^{ef} \cdots Q_l^{ef})S\\ &=& R \cap (Q_1 \cdots Q_l)^{ef}S \\ &=& R \cap P^{f}S \\ &=& P^{f}\end{eqnarray*}
\qed

\item[14. (c)] Let $I$ be an ideal of $S$.  Show $\prod_{\sigma \in G} \sigma(I) = (\norm^{L}_{K}(I))S$.

Let $I = Q_1 \cdots Q_r S$; then $\prod_{\sigma \in G} \sigma(I) = \prod_{\sigma \in G} \sigma(Q_1)\cdots \sigma(Q_r)S$.  With the product taken over all $\sigma \in Q$, $\prod \sigma(Q_i) = P_i$ for some prime ideal $P_i$ of $R$ lying under $I$; therefore $\prod_{\sigma \in G} \sigma(I) = P_1 \cdots P_r S = \norm^{L}_{K}(I)S$.

\item[14. (d)] \begin{eqnarray*}\norm^{L}_{K}(IJ) &=& R \cap \prod_{\sigma \in G} \sigma(IJ) \\&=& R \cap \prod_{\sigma \in G} \sigma(IJ)\\ &=& R \cap \prod_{\sigma \in G} \sigma(I) \prod_{\sigma \in G} \sigma(J) \\ &=& R \cap (\norm^{L}_{K}(I)\norm^{L}_{K}(J)) \\ &=& \norm^{L}_{K}(I)\norm^{L}_{K}(J)\end{eqnarray*}
The final equality holds since $\norm^{L}_{K}(I)$ and $\norm^{L}_{K}(J)$ are ideals in $R$.

\item[14. (e)] If $\beta \in \norm^{L}_{K}((\alpha))$, then $\beta = \sigma_1(\alpha) \cdots \sigma_k(\alpha) \gamma = \norm^{L}_{K}(\alpha)\gamma$; since $\beta \in R$ and $\norm^{L}_{K}(\alpha) \in R$, $\gamma$ must also be in $R$.  Thus $\norm^{L}_{K}((\alpha))$ is the ideal generated by $\norm^{L}_{K}(\alpha)$.

\item[15. (a)] Show for three fields $K \subset L \subset M$, that $\norm^{M}_{K}(I) = \norm^{L}_{K} \norm^{M}_{L}(I)$ for an ideal $I \subset \ringofintegers{M}$.

We show the result for a prime $U$ of $T = \ringofintegers{M}$.  Let $R, S$ be the ring of integers of $K, L, M$ respectively, and let $P$ and $Q$ be the primes of $R$ and $S$ lying under $U$.  Then using the multiplicativity of towers as shown in exercise 10, \[ \norm^{M}_{K}(U) = P^{f(U|P)} = P^{f(U|S)f(S|P)} = \norm^{L}_{K} \norm^{M}_{L}(U) \]

If $I = U_1 \cdots U_r$, then \[ \norm^{M}_{K}(I) = \prod^{r}_{i = 0} \norm^{M}_{K}(U_{i}) = \prod^{r}_{i = 0} \norm^{L}_{K}\norm^{M}_{L}(U_{i}) = \norm^{L}_{K} \norm^{M}_{L}(I) \]

\item[15. (b)] Let $K \subset L$, where $L$ is not necessarily normal.  Extend $L$ to a normal extension $M$.  Let $[M : L] = n$.  We then have:\begin{align*}\norm^{M}_{K}((\alpha)) &= (\norm^{M}_{K}(\alpha)) & \text{(exercise 14. (e))} \\ &= (\norm^{L}_{K}(\norm^{M}_{L}(\alpha))) & \text{Definition of relative norm} \\ &= (\norm^{L}_{K}(\alpha^n)) & \text{$\alpha \in L$ and $L \subset M$} \\ &= (N^{L}_{K}(\alpha))^n & \text{Factorization of ideals} \end{align*}
We also have the following transformation on the norm ideal of $M$ over $K$: \begin{align*}
    \norm^{M}_{K}((\alpha)) &= \norm^{L}_{K} \norm^{M}_{L}((\alpha)) & \text{part (a)} \\
    &= \norm^{L}_{K} ((\alpha^n)) & \text{Exercise 14. (e), $M$ is normal over $L$}  \\
    &= \norm^{L}_{K} ((\alpha)^n) & \text{Factorization of ideals}\\
    &= \norm^{L}_{K} ((\alpha))^n & \text{Exercise 14. (d)}
\end{align*}
We therefore have \[ (\norm^{L}_{K}(\alpha))^n = \norm^{L}_{K}((\alpha))^n \] and conclude that $\norm^{L}_{K}(\alpha) = \norm^{L}_{K}((\alpha))$.

\item[15. (c)]  For the case where $K = \Q$, show that $\norm^{L}_{\Q}(I)$ is the principal ideal in $\Z$ generated by the number $||I||$.

For a prime $Q$ of $L$ lying over a prime ideal $P \subset \Z$ containing the prime $p \in P$, $\norm^{L}_{\Q}(Q) = P^f$, and $||I|| = |R/Q| = p^f$.  Next, suppose $I = Q_1 \cdots Q_r$ where $Q_i$ lies over a prime $P_i$.  By Theorem 22 (a), $||I|| = \prod_{i=1}^{r} ||Q_i|| = \prod_{i=1}^{r} (P_i)^{f(Q_i | P_i)}$, this number then generates the principal ideal $\norm^{L}_{\Q}(I) = \prod_{i=1}^{r} \norm^{L}_{\Q}(Q_i) = \prod_{i=1}^{r} (P_i)^{f(Q_i | P_i)}$.

\item[16.] Let $K$ and $L$ be number fields, $K \subset L$, $R = \ringofintegers{K}, S = \ringofintegers{L}$.  Denote by $G(R)$ and $G(S)$ the ideal class groups of $R$ and $S$ respectively.
\item[16. (a)] Show that the mapping $\psi : G(S) \to G(R)$ defined by taking any $I$ in a given class $C$ and sending $C$ to the class containing $\norm^{L}_{K}(I)$ is a homomorphism.

We first show that $\psi$ homomorphism is well-defined.  Take $I, J \in C$, so there is some element $\alpha, \beta$ such that $\alpha I = \beta J$.  Therefore \begin{eqnarray*}
    \norm^{L}_{K}(\alpha I) &=& \norm^{L}_{K}(\beta J) \\
    \norm^{L}_{K}((\alpha))\norm^{L}_{K}(I) &=& \norm^{L}_{K}((\beta))\norm^{L}_{K}(J) \\
    \norm^{L}_{K}(\alpha)\norm^{L}_{K}(I) &=& \norm^{L}_{K}(\beta)\norm^{L}_{K}(J)
\end{eqnarray*}
Therefore the image of $I$ and $J$ are in the same ideal class, $\psi$ does not depend on the choice of ideal in the class $C$.

$\psi((\alpha)) = N^{L}_{K}((\alpha))$ and so the identity element of the class group maps to the identity element.  $\psi(IJ) = \norm^{L}_{K}(IJ) = \norm^{L}_{K}(I)\norm^{L}_{K}(J)$ and so the mapping respects operation.  Therefore it is a homomorphism.

\item[16. (b)] Let $Q$ be a prime of $S$ lying over a prime $P$ of $R$.  Let $d_{Q}$ denote the order of the class containing $Q$ in $G(S)$, $d_{P}$ denote the order of the class containing $P$ in $G(R)$.  Prove that $d_{P} \mid d_{Q} f$, where $f = f(Q|P)$.

Take $\psi : G(S) \to G(R)$ be the homomorphism defined in 1.  Then $|\psi(Q)| \mid |Q|$. $\psi(Q) = P^{f}$; if $f \mid d_{P}$, $|\psi(Q)| = d_{P} / f$; otherwise $|\psi(Q)| = d_{P}$.  In both cases we have $d_{P} \mid d_{Q} f$.

\item[17.] Let $K = \Q[\sqrt{23}], L = \Q[\w]$, where $\w = e^{2\pi i /23}$.  Let $P$ be one of the primes of $K$ lying over 2; take $P = (2, \theta)$ where $\theta = (1 + \sqrt{-23})/2$, and let $Q$ a prime of $\Q[\w]$ lying over $P$.

\item[17. (a)]] By Theorem 25, $f(Q|2)$ is the multiplicative order of 2 mod 23; $2^{11} = \modequiv{2048}{1}{23}$.  Since $f(P|2) = 1$ ($ref = [K : \Q] = 2$ and $r = 2$) and $f$ is multiplicative in towers, $f(Q|P) = 11$.
\item[17. (b)] $P^3 = (\theta - 2)$: \[ P = (2, \theta), P^2 = (4, 2\theta, \theta - 6) = (4, \theta + 2) \] and \[ P^3 = (8, 4\theta, 2\theta + 4, 3(\theta - 2)) = (\theta - 2) \]  First $\theta - 2 \in P^3: 4\theta - 3(\theta - 2) - 8 = \theta - 2$.  Then, we have $8 = (\theta - 2)(-\theta - 1)$, $4\theta$ = $4(\theta - 2) + 8$, $2\theta + 4 = 2(\theta - 2) + 8$, so every element of $P^3$ is representable as $\theta - 2$ and this is a principal ideal.

However, $P$ is not principal: since $(2, \theta)(2, \overline{\theta}) = (2)$ and the norm of $(2)$ is 4, the ideal $(2, \theta)$ must have norm 2.  For it to be generated by a single $\alpha$ we would need some $(a + b\sqrt{-23})/2 \in \Z[\theta]$ where $a^2 + 23b^2 = 8$.  This has no integer solution so $(2, \theta)$ is not a principal ideal.

Since $P^3$ is a principal ideal the ideal class group of $\Q[\sqrt{-23}]$ must have an order dividing 3.

\item[17. (c)] By 16. (b), the order of $P$ divides the order of $Q$ multiplied by $f(Q|P)$; therefore $3 \mid d_{Q} 11$ and so $3 \mid d_{Q}$.  Therefore $Q$ must also not be a principal ideal.

\item[17. (d)] Suppose $2 = \alpha \beta$ in $\Z[\w]$ and neither $\alpha$ nor $\beta$ is a unit, therefore $2\Z[\w] = (\alpha)(\beta) = (2, \theta)(2, \overline{\theta})$.  By the uniqueness of ideal factorization, $(2, \theta)$ must be principal; however, we have seen that this is not the case in part (c).  This is a contradiction; therefore either $\alpha$ or $\beta$ must be a unit.

\item [18. (a)]  Show $\disc{r\alpha_1, \alpha_2, \ldots, \alpha_n} = r^2 \disc{\alpha_1, \ldots, \alpha_n}$.

Writing the discriminant as the determinant of each of the $\sigma_j$ conjugates of $\alpha_n$, we have:
\[ \disc{r\alpha_1, \alpha_2, \ldots, \alpha_n} = \left|\ \begin{matrix}
    \sigma_1(r\alpha_1) & \cdots & \sigma_k(\alpha_n)
    \\
    \sigma_2(r\alpha_1) & \cdots & \sigma_k(\alpha_n)
    \\
    \vdots & \ddots & \vdots \\
    \sigma_k(r\alpha_1) & \cdots & \sigma_k(\alpha_n) \\
\end{matrix}\ \right|^2 \]
Let $A_{ij}$ be the matrix minor corresponding to row $i$, column $j$.  Since $r \in \Q$, $\sigma_k(r\alpha_1) = r\sigma_k(\alpha_1)$ for all $k$.  Taking the determinant along the first column, we have:
\begin{eqnarray*}
    \disc{r\alpha_1, \alpha_2, \ldots, \alpha_n} &=& \left(\sum_{i = 0}^{n} (-1)^i \sigma_i(r\alpha_1) A_{1i}\right)^2\\
    &=& \left(\sum_{i = 0}^{n} (-1)^i r\sigma_i(\alpha_1) A_{1i}\right)^2\\
    &=& r^2 \left(\sum_{i = 0}^{n} (-1)^i \sigma_i(\alpha_1) A_{1i}\right)^2\\
    &=& r^2 \disc{\alpha_1, \ldots, \alpha_n}
\end{eqnarray*}

\item[18. (b)] Let $\beta$ be a linear combination of $\alpha_2, \ldots, \alpha_n$ with coefficients in $\Q$.  Show $\disc{\alpha_1 + \beta, \alpha_2, \ldots, \alpha_n} = \disc{\alpha_1, \ldots, \alpha_n}$.

For all $\sigma_k$, $\sigma_k(\alpha_1 + \beta) = \sigma_k(\alpha_1) + \sigma_k(\beta)$.  If $\beta = p_2 \alpha_2 + \ldots + p_n \alpha_n$, then $\sigma_k(\beta) = p_2\sigma_k(\alpha_2) + \ldots + p_n\sigma_k(\alpha_n)$ for $p_i \in \Q$.  Writing $\disc{\alpha_1 + \beta, \alpha_2, \ldots, \alpha_n}$ in matrix form, the $k$-th row of the first column has the form $\sigma_k(\alpha_1) + p_2\sigma_k(\alpha_2) + \ldots + p_n\sigma_k(\alpha_n)$.

Subtracting a column times a linear factor has no effect on the determinant of the matrix, so by subtracting $p_i$ multiplied by column $i$ from the first column for each $i$, we see $\disc{\alpha_1 + \beta, \alpha_2, \ldots, \alpha_n} = \disc{\alpha_1, \ldots, \alpha_n}$.

\item[19.] Let $K$ and $L$ be number fields, $K \subset L$, and let $R = \ringofintegers{K}$, $S = \ringofintegers{L}$.  Let $P$ be a prime of $R$.

\item[19. (a)] Show that if $\alpha \in S$, $\beta \in R$, and $\alpha \beta \in PS$, then either $\alpha \in PS$ or $\beta \in P$.

Let $\psi : S / PS \to P / R$ be the homomorphism defined by taking a coset in $S / PS$ to the corresponding coset in $P / R$.  If $\alpha\beta \in PS$, then $\psi(\alpha\beta) = 0$.  Since $P$ is maximal, $R / P$ is an integral domain, so as $\psi$ is a homomorphism into $R / P$ either $\psi(\alpha) = 0$, or $\psi(\beta) = 0$; equivalently, $\alpha \in PS$ or $\beta \in P$.

\item[22. ($\alpha^5 = 2\alpha + 2$)]

Let $\alpha^5 = 2\alpha + 2$ and $R = \ringofintegers{\Z[\alpha]}$.  By Exercise 43, $\disc{\alpha} = 4^4 \cdot (-2)^5 + 5^5 2^4 = 2^4 \cdot 3 \cdot 13 \cdot 67$.

As 2 is the prime with power greater than 1 dividing $\disc{\alpha}$,  we focus on its factorization.  (If $\disc{\alpha}$ we not the whole ring of integers, the order $|R / \Z[\alpha]|$ would be divisible by a power of 2.)

By 43 (d), $\alpha + 1$ is a unit, so we can determine the factorization of 2 by factoring the minimum polynomial of $\alpha + 1$ over $\Q$ mod 2.  Sage gives this as $x^5 - 5x^4 + 10x^3 - 10x^2 + 3x - 1$; mod 2 this is $x^5 + x^4 + x + 1 = (x - 1)^5$; therefore the prime 2 has the factorization $2R = (2, \alpha)^5$ and the inertial degree of the primes lying over 2 is 1.  Using the improvement of Theorem 24, $2^4 \mid \disc{R}$; therefore $\disc{R} = \disc{\alpha}$ and $\ringofintegers{\Z[\alpha]} = \Z[\alpha]$.

\item[22. ($\alpha^5 = 2\alpha^4 + 2$)]

We proceed in a similar method to the previous case: we calculate the discriminant of $\alpha$, take the prime factors that have square terms, and factor the minimum polynomial of $\alpha^4 + 1$ (a unit by 44 (d)) modulo those prime factors.

By 44 (a), $\disc{\alpha} = (-2)^3 (4^4 (-2)^5 + 5^5 (-2)) = 2^4 \cdot 3 \cdot 29 \cdot 83$, so again we only need to focus on $p = 2$.  Sage gives the minimum polynomial of $\alpha^4 + 1$ as $x^5 - 21x^4 + 10x^3 - 10x^2 + 5x - 1$; mod 2 this is $x^5 + x^4 + x + 1 = (x - 1)^5$, so $2R = (2, \alpha^4)^5$ and so $\sum f_i = 1$ and by the improvement of Theorem 24, $2^4 \mid \disc{R}$.  Therefore $\disc{\alpha} = \disc{R}$ and $\ringofintegers{\Z[\alpha]} = \Z[\alpha]$.

\item[24.] Let $R, K, S, L$ be as usual.  A prime $P \subset R$ is {\it totally ramified} if $PS = Q^{n}$, $n = [L : K]$.

\item[24. (a)] Suppose $P$ is totally ramified in $S$; then $PS = Q^{n}$.  Let $M$ be an extension field such that $K \subset M \subset L$ with $\ringofintegers{M} = T$. and $U$ be a prime of $M$ lying over $P$.  Then $U \subset Q$ and $US = Q^{[M : L]}$.  Since the ramification degree is multiplicative in towers, $[L : K] = e(Q|P) = e(Q|U)e(U|P) = [L:M]e(U|P)$; therefore $e(U|P) = [M:K]$ and so $P$ is totally ramified in $M$.

\item[24. (b)] If $P$ is totally ramified in some extension of $L$ and unramified in $L'$, then take $L \cap L'$  By (a), if $L \cap L' \subset L$ then $L \cap L'$ must be totally ramified.  However $L \cap L' \subset L'$ and so must be unramified by assumption.  We conclude $[L \cap L' : K] = 1$ so $L \cap L' = K$.

\item[24. (c)] Let $m = p_1^{e_1} \cdots p_r^{e_r}$.  We prove $[\Q[\w] : \Q] = \phi(m)$ by induction on $r$; TODO

\item[26.]  Let $\alpha = \sqrt[3]{m}$ where $m$ is a cubefree integers, $K = \Q[\alpha]$, $R = \ringofintegers{K}$.
\item[26. (a)] Let $p$ be a prime $\neq 3$ and $p^2 \nmid m$.  By Exercise 2.41, $\disc{\alpha} = -27m^2$, so $p \nmid \disc{\alpha}$.  Therefore $p \nmid |R / R[\alpha]|$ and so the prime decomposition of $p$ in $R$ is determined by factoring the polynomial $x^3 - m \mod p$.
\item[26. (b)] Let $p \neq 3$ and suppose $p^2 \mid m$ and write $m = hk^2$.  We set $\gamma = \alpha^2 / k$.  Note $\gamma^2 = h\alpha$.

By Exercise 2.41, There are two possible integral bases for $R$: either $\{1, \alpha, \alpha^2 / k\}$ ($\modnotequiv{m}{\pm 1}{9}$) or $\{1, \alpha, (\alpha^2 \pm k^2 + k^2)/3k\}$ ($\modequiv{m}{\pm 1}{9}$).

$|R / R[\gamma]| = h$ in the first scenario, $|R / R[\gamma]| = 3h$ in the second.  $h$ is squarefree and so $p \nmid |R/R[\gamma]|$ and so the prime decomposition of $p$ is determined by factoring the minimal polynomial for $\gamma$ mod $p$.  $\gamma^3 = \alpha^6 / k^3 = m^2 / k^3 = h^2k$ so the minimal polynomial for $\gamma$ is $x^3 - h^2 k$.

Since $p \mid k$, this reduces to factoring the equation $x^3$ mod $p$ and so there is one prime lying over $p$ with a ramification degree of 3; therefore $pR = (p, \gamma)^3 = (p, \alpha^2 / k)^3$.

\item[26. (c)] If $\modnotequiv{m}{\pm 1}{9}$, the integral basis for $R$ is $\{1, \alpha, \alpha^2 / k\}$ and so $|R / R[\gamma]| = h$.  We split into two cases, one where $3 \mid k$, one where $3 \nmid k$.

{\bf Case 1: $3 \mid h$}: If $3 \mid h$ then $(3, \alpha)^3$ has generators $(27, 3\alpha^2, 9\alpha, m)$.  Because $m$ is cubefree, $9 \nmid m$, so $\gcd(m, 27) = 3 \in (3, \alpha)$.  Therefore $3R = (3, \alpha)^3$.

{\bf Case 2: $3 \nmid h$}: By Theorem 27, the prime decomposition of $3R$ can be determined by factoring $x^3 - h^2k \mod 3$; mod 3, $x^3 - h^2k \equiv (x - h^2k)^3\ (3)$ and so 3 is totally ramified in $R$ with $3R = (3, \gamma - h^2k)^3$.

\item[26. (d)]  If $m = 10$, then the integral basis for $R$ is $\{1, \alpha, (\alpha^2 + \alpha + 1)/3\}$.  Taking $\beta = (\alpha - 1)^2 / 3$ we note $(\alpha^2 + \alpha + 1) / 3 - \alpha = \beta$ and $\beta^2 = 2(\alpha^2 + \alpha + 1) / 3 - 5$.

Therefore, we have the series of equivalences (using the transformations developed in Exercise 3.18):
\begin{eqnarray*}
    \disc{R} &=& \disc{1, \alpha, \frac{\alpha^2 + \alpha + 1}{3}}\\
             &=& \disc{1, \frac{\alpha^2 + \alpha + 1}{3} - \alpha, \frac{\alpha^2 + \alpha + 1}{3}}\\
             &=& \frac{1}{4} \disc{1, \beta, \frac{2(\alpha^2 + \alpha + 1)}{3}}\\
    4 \disc{R} &=& \disc{1, \beta, \frac{2(\alpha^2 + \alpha + 1)}{3} - 5}\\
             &=& \disc{1, \beta, \beta^2}
\end{eqnarray*}
By Exercise 2.27, $\disc{\beta} = |R/R[\beta]|^2 \disc{R}$ so we conclude that $|R / R[\beta]| = 2$ and so can apply Theorem 27.

By Exercise 2.41, the minimal polynomial for $\beta$ if $m = 10$ is $x^3 - x^2 + 7x - 3$; mod 3, this reduces to $x(x^2 - 2x + 1) = x(x - 1)^2$.  Therefore by Theorem 27, the prime decomposition of $3R = (3, \beta)(3, \beta - 1)^2$.

(Not done: consider for general $\modequiv{m}{\pm 1}{9}$.)

\item[26. (e)] When $\modequiv{m}{\pm 1}{9}$, an integral basis of $R$ is $\{1, \alpha, (\alpha^2 \pm k^2\alpha + k)/3k\}$; we have $|R / R[\alpha]| = 3h$ and by Exercise 2.27 and 2.41, $\disc{\alpha} = |R / R[\alpha]|^2 \disc{R} = -27m^2$, so $\disc{R} = -3h^2k^2$.  Since $\modequiv{m}{\pm 1}{9}$, $3 \nmid h$ (otherwise $\modequiv{m}{0, 3, 6}{9}$) and $3 \nmid k$ (otherwise $\modequiv{m}{0}{9}$).

By exercise 21, $p^{n - \sum f_i} \mid \disc{R}$ and so $\sum f_i = 2$.  Since $\sum f_i e_i = 3$ we must have at least 2 prime ideals lying over 3.  Since $3 \mid \disc{R}$, by the converse to Theorem 24, $3R$ must be ramified with degree greater than 1.  Since the sum of the inertial degrees is also greater than 1, the only possibility satisfying both conditions is that $3R = P^2 Q$ for prime ideals $P$ and $Q$, with both $P$ and $Q$ having inertial degree 1.

\item[27.] Let $\alpha^5 = 5(\alpha + 1)$.  From Exercise 2.43, we know $\disc{\alpha} = 4^4 (-5)^5 + 5^5 (-5)^4 = 5^5 (5^4 - 4^4) = 5^5 \cdot 3^2 \cdot 41$.  This polynomial has the form in exercise 28 ($p = 5$, $r = 1$), and so by Exercise 3.28 (c), $5^4 \mid \disc{R}$, therefore 3 is the only square prime dividing $|\disc{R} / \disc{R[\alpha]}$ and for all other primes, Theorem 27 applies.

$x^5 - 5x - 5 \equiv x^5 + x + 1 \equiv (x^2 + x + 1)(x^3 + x^2 + 1)\ (2)$, so by Theorem 27, $2R = (2, \alpha^2 + \alpha + 1)(2, \alpha^3 + \alpha^2 + 1)$.

I also worked this problem out using the fact that $\alpha + 1$ as a unit; its minimum polynomial over $\Q$ is $x^5 - 5x^4 + 10x^3 - 10x^2 + 1$ and works for any prime (including $p = 3$).  In particular for $p = 2$, $x^5 - 5x^4 + 10x^3 - 10x^2 + 1 \equiv x^5 + x^4 + 1\ (2)$.  This polynomial splits into two factors $x^2 + x + 1$ and $x^3 + x + 1$ over $\Z_{2}$; therefore $2R = (2, (\alpha + 1)^2 + \alpha)(2, (\alpha + 1)^3 + \alpha) = (2, \alpha^2 + \alpha + 1)(2, \alpha^3 + \alpha^2 + 1)$ which matches the other solution.

\item[28.] Let $f(x) = x^n + a_{n-1}x^{n-1} + \ldots + a_0$ where all $a_i \in \Z$ and let $p$ be a prime divisor of $a_0$ with $p^r$ the exact power of $p$ dividing $a_0$ and suppose all $a_i$ are divisible by $p^r$.  Assume $f$ is irreducible over $\Q$ and let $\alpha$ be a root of $f$.  Let $K = \Q[\alpha], R = \ringofintegers{K}$.
\item[28. (a)] $\alpha^n = -(a_{n-1}\alpha^{n-1} + \ldots + a_0) = p^r(\frac{-a_{n-1}}{p^r}\alpha^{n-1} + \ldots \frac{-a_0}{p^r})$, and let $\beta = \frac{-a_{n-1}}{p^r}\alpha^{n-1} + \ldots + \frac{-a_0}{p^r}$.  Then $(\alpha^n) = (p^r)(\beta)$.

Let $\alpha$ have the factorization $\q_1^{e_1} \cdots \q_m^{e_m}$ in $R$; then $(\q_1^{e_1} \cdots \q_m^{e_m})^n = (p^r)(\beta)$ and so \[ \q_1^{n e_1} \cdots \q_m^{n e_m} = (p^r)(\beta) \]
If $(p)$ is not relatively prime with $(\beta)$, then there is some $\alpha' \in K$ such that $\beta\alpha' = p$; therefore $\beta\alpha' - p = 0$ would give a linear dependence of the basis $\{1, \alpha, \ldots, \alpha^{n-1}\}$ over $\Q$, but this set is linearly independent.  Therefore $(p)$ and $(\beta)$ have mutually exclusive factors in $R$ and so (reordering the $\q_i$ if necessary), \[ (p^r) = \q_{1}^{n e_1} \cdots \q_{k}^{n e_k} = \left(\q_1^{e_1} \cdots \q_k^{e_k}\right)^n \].
Therefore $(p^r)$ is an $n$-th power in $R$.

\item[28. (b)] Given the factorization from part (a), we know $r$ divides $ne_i$ for all $i$; if $(n, r) = 1$, then $r$ must divide each of the $e_i$.  Since $p^r$ is an $nr$-th power, $p$ is an $n$-th power.

Since the primes lying over $p$ must have $ref = 1$, we conclude in the factorization of $(p^r)$ must have $e_i = r$ and $f = 1$ and so we have the factorizations $(p^r) = \left(\q^r\right)^n$ and $(p) = \left(\q\right)^n$; therefore $p$ is totally ramified in $R$.

\item[28. (c)] If $r$ is relatively prime to $n$, $p$ is totally ramified in $R$ and so $\sum f_i = 1$ and thus by Exercise 21 (b), $p^{n - 1} \mid \disc{R}$.

We now examine the scenario where $\gcd(n, r) = m$ and take the factorization from part (a).  As in (b), we know $r$ must divide $ne_i$ for all $i$, and so $\frac{r}{m}$ divides $\frac{n}{m}e_i$ for each of the $e_i$.  Since $(p^r)$ is an $n$-th power, then
\begin{eqnarray*}
(p)^r &=& \left(\q_1^{e_1} \cdots \q_k^{e_k}\right)^n \\
(p)^{r / m} &=& \left(\q_1^{e_1} \cdots \q_k^{e_k}\right)^{n / m}\\
(p) &=& \left(\q_1^{\frac{m e_1}{r}} \cdots \q_k^{\frac{m e_k}{r}}\right)^{n / m}
\end{eqnarray*}
Therefore, if $d \neq n$, $(p)$ is ramified with ramification degree at least $n / d$.

We know that for any prime of $\Z$, $ref = n$; therefore \[\sum_{i = 0}^{k} \frac{n}{m}\frac{m e_i f_i}{r} = \frac{n}{m} \sum_{i = 0}^{k} \frac{m e_i f_i}{r}\] and so we must have $\sum_{i = 0}^{k} \frac{m e_i f_i}{r} = m$.  Each of the terms are integers and so $\sum f_i \leq m$; therefore by applying Exercise 21 (b), we have $p^{n - m} \mid \disc{R}$.

This bound is as good as possible.  Let $K = \Q[\alpha]$ where $\alpha$ is a root to the irreducible polynomial $x^4 + 3^2$ and let $R = \ringofintegers{K}$.  $\disc{K} = 2^8 \cdot 3^2$, so $3^2$ is the greatest power dividing the discriminant ($2 = 4 - \gcd(4, 2)$).  The prime $3$ has the factorization $3R = (\alpha)^2$, and so the inertial degree of $(\alpha) = 2$.

\item[28. (d)]
    In both 43 (c) and 44 (d) we have $\alpha$ a root of a degree 5 polynomial satisfying the conditions of 28 (a) with the $a_0$ coefficient $ = a$ where $a$ is squarefree.

    For both equations, we have $p \mid a$, by (c) that $p^4 \mid \disc{R}$.  We have shown for both that $d_3 d_4 \mid a^2$, and we know $d_3 \mid d_4$.

    {\bf 43 (c):} $\disc{\alpha} = a^4(4^4 a + 5^5) = (d_3 d_4)^2 \disc{R}$.
    By assumption $4^4a + 5^5$ is squarefree.  Suppose $p \mid d_3$ or $p \mid d_4$; then $p^6 \mid (d_3 d_4)^2\disc{R}$.  This implies $a^2 \mid 4^4a + 5^5$, contradicting $4^4a + 5^5$ squarefree.

    {\bf 44 (d):} $\disc{\alpha} = a^4[(4a)^4 + 5^5] = (d_3 d_4)^2 \disc{R}$.  As in the previous case, $p^6 \mid (d_3 d_4)^2 \disc{R}$ and so $a^2 \mid (4a)^4 + 5^5$, contradicting the assumption that this quantity is squarefree.

\item[29.]
    Let $\alpha$ be an algebraic integer and let $f$ be a monic irreducible polynomial for $\alpha$ over $\Z$.  Let $R = \ringofintegers{\Q[\alpha]$ and suppose $p$ is a prime in $\Z$ such that $f$ has a root $r$ in $\Z_p$ and $p \nmid |R/\Z[\alpha]}$.

\item[29. (a)] Show there is a ring homomorphism $R \to \Z_p$ that takes $\alpha$ to $r$.

    Since $f(r) \equiv 0\ (p)$ and $p \nmid |R/\Z[\alpha]|$, by Theorem 27, the prime ideal $Q = (p, \alpha - r)$ lies over $P$.  As $x - r$ is a factor of $f(x)$ mod $p$, the inertial degree of $Q$ is 1, and so $|R / Q| = |p|$, so $R / Q \simeq \Z_{p}$.  Let $\psi$ be the mapping from $R$ to its quotient ring $R / Q$: since $\alpha - r \in Q$, $\psi(\alpha) = r$.

\item[29. (b)] Let $\alpha^3 = \alpha + 1$.  Show $\sqrt{\alpha} \not\in \Q[\alpha]$.

    By exercise 2.28, $\ringofintegers{\Q[\alpha]} = \Z[\alpha]$, so $|\Z[\alpha]/\Z[\alpha]| = 1$.  Since $\Z[\alpha]$ is integrally closed in $\Q[\alpha]$ it suffices to show $\sqrt{\alpha} \not\in \Z[\alpha]$: we will do this by finding appropriate $r, p$ such that $r$ is a root of $x^3 - x - 1$ mod $p$ and $r$ is not a square mod p.

    As suggested in the hint, we take $r = 2$ and $p = 5$.  $2^3 - 2 - 1 \equiv 0\ (5)$ and there is a ring homomorphism $\psi$ from $\Z[\alpha] \to \Z_5$ where $\psi(\alpha) = 2$.  If $\sqrt{\alpha} \in \Z[\alpha]$, then $\psi(\sqrt{\alpha})^2 = 2$; however, $2$ is not a square mod 5.  Therefore $\sqrt{\alpha} \not\in \Z[\alpha]$ and so $\sqrt{\alpha} \not\in \Q[\alpha]$.

\item[29. (c)] Show $\sqrt[3]{\alpha}$ and $\sqrt{\alpha + 2}$ are not in $\Q[\alpha]$.

{\bf $\sqrt[3]{\alpha} \not\in \Q[\alpha]$}:  Let $r = 5$; then $5^3 - 5 - 1 = 119 \equiv 0\ (7)$; however there is no element such that $x^3 \equiv 5\ (7)$.  Therefore $\sqrt[3]{\alpha} \not\in \Q[\alpha]$.

{\bf $\sqrt{\alpha + 2}$}:  Let $r = 3$; then $3^3 - 3 - 1 = 23 \equiv 0\ (23)$; however $5$ is not a quadratic residue mod 23.  Therefore $\sqrt{\alpha + 2} \not\in \Q[\alpha]$.

\item[29. (d)] Let $\alpha^5 + 2\alpha = 2$.  Prove $x^4 + y^4 + z^4 = \alpha$ has no solutions in $\ringofintegers{\Q[\alpha]}$.  By exercise 2.43, $\disc{\alpha} = 4^4 (2)^5 + 5^5 (-2)^4 = 58192 = 2^4 * 3637$ so all primes except $2$ and $3637$ satisfy $|\ringofintegers{Q[\alpha]} : \Z[\alpha]|$.

Taking $r = 4$ and $p = 5$, we see $4^5 + 2\cdot 4 - 2 = 130 \equiv 0\ (5)$.  Letting $\psi$ be the homomorphism from (a), we observe that if there were $x, y, z$ such that $x^4 + y^4 + z^4 = \alpha$, we would have $\psi(x^4 + y^4 + z^4) = \psi(\alpha) = 4$.  However in $\Z_5$, $x^4 \equiv 1$ for all $x$ so $\psi(x^4 + y^4 + z^4) = \psi(x)^4 + \psi(y)^4 + \psi(z)^4 = 3 \neq 4$.  Therefore there are no $x, y, z \in \Q[\alpha]$ such that $x^4 + y^4 + z^4 = \alpha$.

\item[30. (a)] Let $f$ a nonconstant polynomial $f(x)$ over $\Z$ with $f(0) = 1$.  Suppose there are only a finite number of primes such that $\modequiv{f(x)}{0}{p}$ has a root; then there must be some largest prime $P'$ for which a root exists.  Consider the prime divisors of $f(P'!)$: for every prime $p$, $\modequiv{f(P'!)}{1}{p}$, so it must have a prime divisor $q > P'$.  However, then $\modequiv{f(P'!)}{0}{q}$ contradicting that there were only a finite number of primes such that $f$ had a root.

Next, suppose $f(x)$ is a nonconstant polynomial over $\Z$.  If $f(0) = 0$ then $f(0) = 0$ and so $f$ has a root for all primes $p$.  Suppose $f(0)$ is nonzero, then the polynomial $g(x) = f(f(0)x) / f(0)$ is also in $\Z$ (as $f(0)$ must divide each coefficient).  $\modequiv{g(x)f(0)}{f(f(0)x)}{p}$; as $\Z[x]$ is an integral domain, $g(x)$ has a root mod p if and only if $f(x)$ has a root.  As $g(0) = 1$ it has a root for infinitely many primes and so does $f(x)$.

\item[30. (b)]  Let $K = \Q[\alpha]$ be a number field and take $f(x)$ be the minimal polynomial for $\alpha$.  Let $R = \ringofintegers{K}$ and consider the value $|R : \Z[\alpha]|$.  For any prime $p \nmid |R : \Z[\alpha]|$ such that $f(x)$ has a root $r$ mod $p$, by Theorem 27, there is a prime ideal of the form $(p, \alpha - r)$ lying above $P$.  The inertial degree of this prime ideal is 1 (as $x - r$ has degree one).  As $|R: \Z[\alpha]|$ is a finite value, only finitely many primes divide it.  However $f$ has a root for infinitely many primes $p$.  Therefore there are infinitely many primes $p$ such that $f(P|p) = 1$.

\item[30. (c)]  Take the polynomial $x^m - 1$.  If for any prime $\modequiv{x^m - 1}{0}{p}$ has a solution, then $\modequiv{x^m}{1}{p}$ and so $m \mid p - 1$; thus there exists $k$ such that $km = p - 1$ and so $\modequiv{1}{p}{m}$.  By (a) the polynomial $x^m - 1$ has roots for infinitely many primes $p$, so for any $m$, an infinite number of primes $p$ exist such that $\modequiv{p}{1}{m}$.

\item[30. (d)] Let $L$ and $K$ be number fields.  Take $M$ the normal closure of $K$.  Only finitely many primes are ramify in $M$; however by (b) there are an infinite number of primes $p$ such that $f(Q|p) = 1$ for $Q$ a prime ideal lying over $p$.  Taking away the primes that ramify, in $M$ these primes have inertial degree 1 and ramification index 1, so they must split completely.   As $L$ is an intermediate field and inertial degree/ramificiation index is multiplicative, these primes must also have inertial degree/ramificiation index equal to 1 in $L$.  Therefore there are an infinite number of primes $p$ that split into $[L : K]$ distinct factors in the intermediate field $L$.

\item[30. (e)] TODO

\item[31. (a)] For fractional ideals $A, B$, let $A = \alpha I$ and $B = \beta I$.  For $r \in A, s \in B$, then $r = \alpha i$ and $s = \beta j$ where $i \in I, j \in J$.  Therefore $rs = \alpha i \beta j = \alpha\beta ij \in \alpha\beta IJ$.  Conversely assume $c \in \alpha\beta IJ$ then $c = \alpha\beta c'$ where $c' \in IJ$, so $c'$ has the form $rs$ for $i \in I$, $j \in J$.  Therefore $c = \alpha i \beta j$ and so is a member of $\alpha I \beta J = AB$.  Therefore the product of fractional ideals is independent of the representation of its factors.

\item[31. (b)] Let $A = \alpha I$ for $\alpha \in K$, $I \subset R$; we will show $A^{-1} A = R$.  By Theorem 15 there is some $J$ such that $IJ$ is principal, generated by some $\beta \in R$.

{\bf Claim: $A^{-1} = \alpha^{-1}\beta^{-1}J$}

Take $a \in A^{-1}$. We have the following series of inclusions:
\begin{eqnarray*}
    aA = a\alpha I &\subset& R \\
    a\alpha I J &\subset& RJ = J\\
    a \alpha (\beta) &\subset& J\\
    (a) &\subset& \alpha^{-1}\beta^{-1} J
\end{eqnarray*}
Therefore $a \in \alpha^{-1}\beta^{-1} J$.  Conversely $\alpha^{-1}\beta^{-1} J \subset A^{-1}$ as $\alpha^{-1}\beta^{-1}IJ = (1) \subset R$.  This proves the claim.

Using the claim, $A A^{-1} = \alpha^{-1}\alpha \beta^{-1} IJ = \beta^{-1} (\beta) = (1) = R$.

\item[31. (c)] Let $A$ be a fractional ideal of the form $\alpha I$ for $\alpha \in K$, $I \subset R$.  As $K$ is the field of fractions of $R$, $\alpha = r / s$ for some $r \in R$, $s \in S$.

Let $I$ have the factorization into prime ideals $P_1^{e_1} \cdots P_k^{e_k}$, $(r)$ have factorization $P_{k+1}^{e_{k+1}} \cdots P_{m}^{e_m}$, and $(s)$ have factorization $P_{m+1}^{e_{m+1}} \cdots P_{r}^{e_r}$; combining the terms and removing the primes raised to the 0th power, gives the prime factorization of $A$ as \[ A = P_1^{e_1} \cdots P_k^{e_k} P_{k+1}^{e_{k+1}} \cdots P_{m}^{e_m} P_{m+1}^{-e_{m+1}} \cdots P_{r}^{-e_r} \]
Given two different factorizations of $A$, $A = P_1^{e_1} \cdots P_k^{e_k} = Q_1^{e'_1} \cdots Q_j^{e'_j}$, we have $R = (P_1^{e_1} \cdots P_k^{e_k})^{-1} Q_1^{e'_1} \cdots Q_j^{e'_j} = (P_1^{-e_1} \cdots P_k^{-e_k}) Q_1^{e'_1} \cdots Q_j^{e'_j}$.  Since this product is equal to $R$, the $P_i$ and $Q_i$ terms must all cancel showing that the factorizations were identical.

\item[31. (d-f)] TODO

\item[34.] Let $\{ \alpha_1, \ldots, \alpha_n \}$ be a basis for $L$ over $K$.
\item[34. (a)] Since the $\alpha_i$s form a basis for $L$ over $K$ their discriminant is nonzero and so the matrix $M$ corresponding to the trace product has nonzero determinant.  Take $\beta_i = \sum_{k = 1}^{n} \alpha_k M^{-1}_{ik}$.  Then $\trace{\beta_i\alpha_j} = \sum_{k = 1}^{n} \alpha_j\alpha_k M^{-1}_{ik} = \sum_{k = 1}^n M_{ki} M^{-1}_{ik} = \delta_{ij}$ where $\delta_{ij}$ is the Kronecker delta function.  Therefore the $\beta_i$ form a dual basis to the $\alpha_i$.
\item[34. (b)]
Let $A = R\alpha_1 \oplus \cdots \oplus R \alpha_n$.  Show that $A^{*} = B$ where $B$ is the $R$-module generated by the $\beta_i$.

As the $\beta_i$ were written as linear combinations of the $\alpha_i$s, we know $B \subset A^{*}$; it remains to show the opposite direction.

Let $\gamma \in A^{*}$, define $m_i = \text{Tr}^{K}_{L}(\gamma\alpha_i)$; by assumption $m_i \in R$.  Take $\beta = \sum_{i=1}^{n} m_i \beta_i$.  For any $\alpha \in A$, $\alpha = r_1 \alpha_1 \oplus r_n \alpha_n$ so $\text{Tr}^{L}_{K}(\gamma\alpha) = \sum_{i = 0}^{n} r_i m_i = \text{Tr}^{L}_{K}(\beta\alpha)$.  Therefore $\text{Tr}^{K}_{L}((\gamma - \beta)A) = 0$.

We claim $\gamma - \beta = 0$.  Since $A$ is a free $R$-module generated by the $\alpha_i$, each $\alpha_i \in A$.  If $(\gamma - \beta)^{-1} \neq 0 \in L$, it can be written as a sum of the $\alpha_i$ with coefficients in $K$.  As $K$ is the field of fractions of $R$ there is some $r \neq 0$ such that $r$ clears the denominators of the coefficients of the $\alpha_i$ and so $r(\gamma - \beta)^{-1} \in A$.  Then $\text{Tr}^{L}_{K}(r(\gamma - \beta)(\gamma - \beta)^{-1}) = \text{Tr}^{L}_{K}(r) = rn$ where $n = [L : K]$.  However $\text{Tr}^{L}_{K}((\gamma - \beta)\alpha) = 0$ for all $\alpha \in A$.  Therefore $\gamma - \beta = 0$ and so $A^{*} \subset B$.  We conclude $A^{*} = B$.

\item[35.] Let $\alpha \in L$, $L = K[\alpha]$.  Let $f$ be the monic irreducible polynomial for $\alpha$ over $K$, and write $f(x) = (x - a)g(x)$.  Then we have \[ g(x) = \gamma_0 + \gamma_1 x + \ldots + \gamma_{n-1}x^{n-1} \]
We claim that \[ \left\{ \frac{\gamma_0}{f'(\alpha)}, \ldots, \frac{\gamma_{n-1}}{f'(\alpha)} \right\} \] is the dual basis to $\{ 1, \alpha, \ldots, \alpha^{n-1} \}$.

\item[35. (a)] Let $\sigma_1, \ldots, \sigma_n$ be the embeddings of $L$ in $\mathbb{C}$ fixing $K$ pointwise.  Then the $\sigma_i(\alpha)$ are the roots of $f$.  Applying $\sigma_i$ to $f(x)$ gives \[\sigma_i(f(x)) = \sigma_i(x - \alpha)\sigma_i(g(x)) = (x - \alpha_i)g_{i}(x)\] where $g_i(x)$ is $g(x)$ with $\sigma_i$ applied to each coefficient and $\alpha_i$ is the conjugate $\sigma_i(\alpha)$.
\item[35. (b)] By the chain rule, we have
\[ f'(x) = (x - \alpha_i)g'_i(x) + g_i(x) \]
So rearranging terms gives \[g_i(\alpha_j) = f'(\alpha_j) - (\alpha_j - \alpha_i)g'_i(\alpha_j) \] Clearly $g_i(\alpha_i) = f'(\alpha_i)$.  If $i \neq j$, $f'(\alpha_j) = (\alpha_j - \alpha_i)g_i'(\alpha_j) + g_i(\alpha_j)$; $g_i(\alpha_j) = 0$ since $g_i$ is a root of all conjugates of $\alpha$ except for $\alpha_i$.  Therefore $g_i(\alpha_j) = 0$.

\item[35. (c)] Let $M$ be the matrix formed by $[ \alpha_j^{i-1}]$ where $i$ is the row and $j$ is the column.  Let $N$ be the matrix $[\sigma_i(\gamma_{j - 1}/f'(\alpha))]$.

Take $NM_{ij}$ as the $i$th row and $j$th column of the matrix product $NM$.  Then $NM_{ij} = \sum^{n}_{k = 1} \alpha_{j}^{k - 1} \sigma_i(\gamma_{k - 1} / f'(\alpha)) = \frac{1}{f'(\alpha)} g'_i(\alpha)$.  By (b) this value is 1 when $i = j$ and 0 otherwise; therefore $NM$ is the identity matrix.

Since $\alpha$ is the root of a monic polynomial over $K$ it must be such that $\alpha \neq K$ and $\disc{\alpha} \neq 0$; therefore the matrix $M$ is invertible and so $NM = I$ implies $N = M^{-1}$; so $MN = I$.  $MN_{ij} = \sum_{k = 1}^{n} \sigma_{k}(\alpha^{i - 1}\gamma_{j - 1}/f'(\alpha)) = \trace{\alpha^{i-1}\gamma_{j - 1}/f'(\alpha)}$.

Since $MN$ is also the identity element the set $\{ \frac{\gamma_{0}}{f'(\alpha)}, \ldots, \frac{\gamma_{n - 1}}{f'(\alpha)} \}$ is therefore the dual basis to $\{ 1, \alpha, \ldots, \alpha^{n-1} \}$.

\item[35. (d)]
Let $a_i$ be the coefficient of the $i$th power of $f(x)$.  Multipliying out $f(x) = (x - \alpha)g(x)$, \[ -\gamma_0\alpha + (\gamma_0 - \gamma_1 \alpha)x + (\gamma_1 - \gamma_2 \alpha)x^2 + \cdots \gamma_{n-1}x^n \]

To show the the $\gamma_i$ as an $R$-module generate $R[\alpha]$ we prove the following lemma by induction:

\begin{lemma}
     For $i \neq n - 1$, $\gamma_i = \sum \alpha^{n - i - 1} + a_{n - i - 1} \alpha^{n - i - 2} + \ldots + a_{i + 1}$.
\end{lemma}
\begin{proof}
From the above multiplication, $-\gamma_0\alpha = a_0 = -(\alpha^n + a_{n-1} \alpha^{n-1} + \ldots + a_1 \alpha)$ and so $\gamma_0 = \alpha^{n-1} + a_{n-1} \alpha^{n-2} + \ldots + a_1$.  This is a sum of powers with leading coefficient 1 and constant term equal to $a_{i}$.  Therefore the base case is satisfied.

Next, assume $\gamma_i = \alpha^{n - i - 1} + a_{n - i - 1}\alpha^{n - i - 2} + \ldots + a_{i + 1}$. From the above expansion we have $a_{i + 1} = (\gamma_{i} - \alpha\gamma_{i + 1})$, so $a_{i + 1} = a_{i + 1} + a_{i + 2}\alpha + \ldots + \alpha^{n - i - 1} - \alpha\gamma_{i + 1}$ and so $\alpha\gamma_{i + 1} = \alpha(\alpha^{n - i - 2} + \ldots + a_{i + 2})$.  Therefore $\gamma_{i + 1}$ can also be written in the appropriate form.
\end{proof}

For $i = n - 1$, $\gamma_{n-1} = 1$ since $f(x)$ is a monic polynomial.   There is then a translation matrix between the powers of $\alpha$ and the $\gamma_i$s with 1s on the diagonal.  This translation matrix is upper triangular since the power of $\alpha$ in row $i$ is $i - 1$.  This matrix is invertible over $\Z$ and the powers of $\alpha$s must also be writable in terms of the $\gamma_i$s.  Therefore the $\gamma_i$s generate $R[\alpha]$ as an the $R$-module.

\item[35. (e)] We have the following: \begin{align*}(R[\alpha])^{*} &= \left\{ \frac{\gamma_0}{f'(\alpha)}, \ldots, \frac{\gamma_{n-1}}{f'(\alpha)}\right\} & \text{By 34. (b)} \\ &= \frac{1}{f'(\alpha)}\left\{ \gamma_0, \ldots, \gamma_{n-1}\right\}\\ &= \frac{1}{f'(\alpha)}R[\alpha] & \text{By 35. (d)} \end{align*}

\item[35. (f)] Let $\beta \in \diff R[\alpha]$; then $\beta (R[\alpha])^{*} \subset S$.  By (e), $\beta(\frac{1}{f'(\alpha)})R[\alpha] \subset S$ so $\beta R[\alpha] \subset f'(\alpha)S$.  Therefore $\beta \in f'(\alpha)S$.  Therefore $\diff R[\alpha] \subset f'(\alpha)S$.  The reverse inclusion is straightforward.  Therefore $\diff R[\alpha] = f'(\alpha)S$.

\item[35. (g)] Let $R[\alpha] \subset S$ so $\diff R[\alpha] = R[\alpha]\ \diff S \subset \diff S$ by the final equality of Exercise 33.  By (f), $\diff R[\alpha] = f'(\alpha) S$.  Since $f'(\alpha) \in \diff f'(\alpha)S$ therefore $f'(\alpha)\in \diff S$.
\end{enumerate}


\section*{Chapter 4}

\begin{enumerate}
    \item [1.] Prove from the definitions that $E(Q|P) \triangleleft D(Q|P)$.

    Let $\sigma \in E(Q|P)$ and $\tau \in D(Q|P)$.  We have the following series of equivalences:
    \begin{eqnarray*}
        \alpha &\equiv& \alpha\mod Q \\
        \tau(\alpha) &\equiv& \tau(\alpha)\mod Q\\
        \sigma(\tau(\alpha)) &\equiv& \tau(\alpha) \mod Q\\
        \tau^-1(\sigma(\tau(\alpha))) &\equiv& \tau^{-1}\tau(\alpha) \equiv \alpha\mod Q
    \end{eqnarray*}
    Therefore $E$ is normal in $D$.

    \item[3. (a)]
    Let $p$ be an odd prime.  Since the multiplicative group of any prime is cyclic, it must have a generator $g$ with order $p - 1$.  As $\modequiv{g^{p-1}}{1}{p}$, $\modequiv{g^{(p-1)/2}}{-1}{p}$.  Therefore if $\modequiv{p}{1}{4}$ then $\modequiv{g^{(p-1)/4})^2}{-1}{p}$ and $\left(\frac{-1}{p}\right) = 1$.  Conversely, if $\left(\frac{-1}{p}\right) = 1$, let $a$ be such that $\modequiv{a^2}{-1}{p}$; therefore $\modequiv{a^4}{1}{p}$ and so $4 \mid p - 1$.

    % Still to show: a/p b /p = ab /p for integers a/b coprime to p.

    \item [5.] Let $K$ and $L$ be number fields, $L$ a normal extension of $K$ with Galois group $G$, and $P$ a prime of $K$.  By "intermediate field" we mean "intermediate field distinct from $K$ and $L$".

    \item [5. (a)] If $P$ is inert in $L$, then as $ref = n$, $f = n$.  The decomposition group $D(Q|P)$ is thus all of $G$.  As $G$ is normal in itself, Corollary 1 to Theorem 28 gives us that $G$ is cyclic of order $n$.
    \item [5. (b)] Suppose $P$ is totally ramified in every intermediate field, but not totally ramified in $L$.  Take $L_E$ to be the inertia field of $P$; by Theorem 28, the ramification index of $L_{E}$ is 1.  By assumption $L_{E}$ is totally ramified and so $[L_{E} : K] = 1$; therefore $L_{E} = K$ and so $f(Q|P) = 1$ and $r(Q|P) = 1$; therefore $P$ must be totally ramified in $L$ also, contradicting our assumption.  Therefore no intermediate fields distinct from $K$ and $L$ must exist.

    Since no intermediate fields exist, $G$ must have no proper subgroups and so be of prime order for some prime $p$, and so must also be cyclic.

    \item [5. (c)] Suppose every intermediate field contains a unique prime lying over $P$ but $L$ does not.  We argue in similar style to (c).  Take $L_D$ to be the decomposition field of $P$.  By Theorem 28, there must only be 1 unique prime lying over $L_D$ and so $[L_D : K] = 1$ and therefore $r = 1$.  Therefore $n = ef$ and so there is one unique prime lying over $P$ in $L$, contradicting our assumption, and so no intermediate fields distinct from $K$ and $L$ exist.  Therefore $G$ is cyclic of prime order as in (b).

    \item [5. (d)] If $P$ is unramified in every intermediate field but ramified in $L$, then in particular $P$ is unramified in $L_{E}$.  Let $H \subset G$, then as $L_{H}$ is unramified, $L_{H} \subset L_{E}$ and so $E \triangleleft H$.  As $L_{E}$ is also an intermediate field and is unramified, $[L : L_{E}] \neq 1$ and so $E$ is nontrivial.  Therefore $E$ is the unique smallest nontrivial subgroup of $G$.

    Since $E$ has no subgroups, it must be of prime order for some prime $p$.  As it is the unique subgroup of order $p$, it must be normal in $G$, as it is the sole element of its conjugacy class.  Since $Z(G) \neq \emptyset$ in a group of prime power order and is a normal subgroup of $G$, $E$ must also contained be $Z(G)$.

    Because every subgroup of $G$ contains $E$, by the Sylow theorems, every subgroup of $G$ must have prime power order, including $G$ itself.  (Otherwise there would be a $q$-subgroup $H$ with $H \cap E \neq \emptyset$ which would give an element of $H$ with order $p$, a contradiction.)

    \item [5. (e)] If $P$ splits completely in every intermediate field but not in $L$, then $P$ must split completely in $L_{D}$ and so $L_{D} \neq K$.

    Let $M$ be an intermediate field of $L$; then $r_{M} = [L : M]$; but $r_{M} \leq r$.  Therefore any intermediate field of $L$ must be a subfield of $L_{D}$ and there are no intermediate fields between $L_{D}$ and $L$.

    Therefore, for any nontrivial subgroup $H \subset G$, $D \subset H$, and $D$ has have no proper subgroups.  Therefore it must be of prime order for some $p$ and so cyclic.  As in (d), $G$ must be a group of order $p^k$, $D \triangleleft G$, and $D \subset Z(G)$.

    {\bf An example over $\Q$:} Let $L$ be the cyclotomic field $\Q[\zeta_5]$.  The Galois group of $L$ has order 4 and is cyclic with generator $\sigma$.

    Let $p = 19$; $\modequiv{19}{-1}{5}$ so $\modequiv{19^2}{1}{5}$ and so 19 has multiplicative order 2 mod 5.  By Theorem 26, its inertial degree in $\Z[\zeta_5]$ is 2 and as $\gcd(5, 19) = 1$, its ramification index is 1; since $ref = 4$, 19 must split into 2 primes in $\Z[\zeta_5]$.

    Because there are 2 primes lying over 19 in $\Z[\zeta_5]$, and $\sigma$ generates the Galois group, $\sigma$ must permute the primes lying over 19, meaning its decomposition group $D = \{e, \sigma^2\}$.  This is a normal subgroup in $G$ and so Corollary 2 to Theorem 28 applies.

    As there are no other subgroups of $G$, 19 splits completely in every proper subfield of $\Q[\zeta_5]$ but not in $\Q[\zeta_5]$, where it has inertial degree 2.

    \item [5. (f)]  Let $P$ be inert in every intermediate field but not inert in $L$.  By (b), for there to be an intermediate field, $P$ must be ramified in $L$ with degree $e$.  By (d), $G$ is a group of prime power order.

    Let $E$ be the inertia subgroup of $P$: since $P$ remains inert in every subgroup, there $E$ is a maximal subgroup in $G$.  Applying (a) to $L_{E}$ we have that $E$ is cyclic; as $E$ is a unique maximal subgroup, $G$ is therefore also cyclic.

    \item[7. (a)] Let $p = 3$, $K = \Q[\sqrt{-3}]$, $L = \Q[\sqrt{3}]$.  Since $\Q[i] \subset KL$ and $p$ is inert in $\Q[i]$, $p$ is not totally ramified in $KL$ despite being totally ramified in $K$ and $L$.

    \item[7. (b)] Let $p = 2$.  By Theorem 25, $p$ ramifies in any quadratic field $\Q[\sqrt{m}]$ with $\modequiv{m}{3}{4}$; however, $p$ splits in any qudaratic field $\modequiv{1}{3}{4}$.  Since $\modequiv{3^2}{1}{4}$, any two extensions of $\Q$ $\modequiv{m, n}{3}{4}$, $\gcd(m, n) = 1$, and $\modnotequiv{mn}{5}{8}$ will contain a subfield where $p = 2$ splits into 2 distinct primes.

    Let $K = \Q[\sqrt{7}]$, $L = \Q[\sqrt{15}]$: then $\modequiv{7 \cdot 15}{1}{8}$ and so $2$ does not remain inert and splits into two primes in its subfield $\Q[\sqrt{105}]$.  Therefore $2$ is splits into two ramified prime ideals in $KL$.

    \item[7. (c)] Let $p = 2$.  By Theorem 25, $p$ is inert in any quadratic field $\Q[\sqrt{m}]$ where $\modequiv{m}{5}{8}$.  Therefore $p$ is inert in both $K = \Q[\sqrt{5}]$ and $L = \Q[\sqrt{13}]$. However, the composite field $KL$ contains the subfield $\Q[\sqrt{65}]$ and $\modequiv{65}{1}{8}$, and by Theorem 25, $2$ splits into two primes in $\Q[\sqrt{65}]$.  Therefore $2$ also splits into two primes in $KL$ each with inertial degree 2.

    \item[7. (d)] Let $p = 2$.  We reverse the procedure we used in (b) and look for any two relatively prime integers $m, n$ such that $\modequiv{mn}{5}{8}$.  Take $m = 3, n = 15$; then $p$ is ramified in both $K = \Q[\sqrt{3}]$ and $L = \Q[\sqrt{15]}$ but remains inert in the subfield $\Q[\sqrt{5}] \subset KL$.  Therefore $p$ has a residue field extension of degree 2 in $KL$.

    \item[8.] Let $r, e, f$ be positive integers.  I will assume Dirichlet's Theorem on primes in arithmetic progression: given integers $a$ and $d$ such that $\gcd(a, d) = 1$, there exists a prime $p$ such that $p = a + nd$.

    \item[8. (a)] The $q$th cyclotomic field has degree $q - 1$ over $\Q$.  Since $q$ is the only prime dividing $\disc{\Z[\zeta_q]}$, no distinct prime $p$ will ramify in $\Q[\zeta_q]$; therefore, by Theorem 26, if $p$ has multiplicative order $f$ mod $q$, it will split into $(q - 1) / f$ distinct primes in $\Q[\zeta_q]$.

    For any given $r$, take $q$ a prime such that $q \equiv 1\ (r)$; by Exercise 30 (c), there are an infinite number of primes that satisfy this property.

    Since $q$ is a prime, it has a primitive root $g$ and $g^{r}$ has order $f$ in $q$.  It remains to find a prime $p$ such that $\modequiv{p}{g^{r}}{q}$.  Such a prime $p$ always exists by Dirichlet's Theorem on primes in arithmetic progression.

    \item[(b)] Take $q$ to be a prime such that $\modequiv{q}{rf}{1}$; then there exists some $k$ such that $krf + 1 = q$.  Let $g$ be a primitive root for $q$: then $q^{r}$ is an element of order $fk$, so using Dirichlet's Theorem we find $p$ such that $\modequiv{p}{g^{r}}{q}$.  Since $p$ has order $fk$ mod $q$, the prime ideal $(p)$ in $\Z$ splits into $r$ distinct prime ideals in $\Q[\zeta_{q}]$.

    As $\text{Gal}(\Q[\zeta_{q}])$ is cyclic, the comment after Theorem 28 applies and $\Q[\zeta_{q}]$ splits into $r$ distinct prime ideals in every subfield containing the decomposition field, which is of order $r$.  Therefore $p$ also splits into $r$ distinct prime ideals in the subfield $K'$ of degree $rf$.

    \item[(c)] When choosing $p$, we apply the Chinese Remainder Theorem to the system of equivalences $\modequiv{g^{kr}}{p}{q}$ and $\modequiv{p}{1}{e}$ (choosing a $q$ such that $\gcd(q, e) = 1$).  This gives an integer $M$, possibly composite, such that $\modequiv{M}{1}{e}$ and $\modequiv{M}{g^{kr}}{q}$.  We know that $\modnotequiv{M}{1}{q}$ since $g$ was chosen to be a primitive root of $q$.  Therefore $\gcd(M, qe) = 1$, and we can apply Dirichlet's Theorem on primes in arithmetic progression to find a prime $p$ such that $p = M + nqe$ for some $n$; therefore $\modequiv{p}{1}{e}$ and $\modequiv{p}{g^{kr}}{q}$.

    \item[(d)] In the $p$th cyclotomic field, $p$ is totally ramified.  Since $\Q[\zeta_p]$ is normal over $\Q$, $p$ is totally ramified in every intermediate field.  Finally, $\text{Gal}(\Q[\zeta_p])$ is cyclic so it has a normal subgroup of order $d$ for each divisor of $p - 1$.  As $p$ was chosen such that $\modequiv{p}{1}{e}$.  It follows that $\Q[\zeta_p]$ has a subfield $K''$ such that $[K'' : \Q] = e$.

    The composition field $K' K'' \subset \Q[\zeta_{q}]$ and has degree $ref$ over $\Q$.  Since $p$ splits into $r$ distinct factors in $K'$ and is ramified in $K''$, it splits into $r$ distinct factors, each with ramificiation index $e$.

    \item[(e)]  Take $e = 2, f = 3, r = 5$.  We start by finding a prime $q$ such that $\modequiv{q}{1}{fr}$; $q = 31$ works (here $k = 2$).  31 has several primitive roots - we want to choose one such that $g^{r} = g^{5}$ is smallest as this will make our search for a prime easiest.  In particular $13$ is a primitive root of 31 and $\modequiv{13^{5}}{6}{31}$, so we choose $p = 37$.

    The $31 \cdot 37 = 1147$th cyclotomic field therefore has a subfield of degree $ref = 30$ over $\Q$ (by (d)) where the prime ideal $(5)$ splits into $5$ distinct primes, each with ramification index $2$.

\item[9.] Let $L$ be a normal extension of $K$, $P$ a prime of $K$, and $Q$ and $Q'$ of $L$ lying over $P$.  Since $L$ is normal there is some $\sigma$ such that $Q' = \sigma Q$.  Let $D$ and $E$ be the decomposition and inertia groups for $Q$ over $P$ and $D'$ and $E'$ the corresponding things for $Q'$ over $P$.

\item[9. (a)] Prove that $D' = \sigma D \sigma^{-1}$ and $E' = \sigma E \sigma^{-1}$.

{\bf $D' = \sigma D \sigma^{-1}$:} Suppose $\tau \in D$; then $\tau(Q) = Q$.  As $Q' = \sigma Q$, $\sigma^{-1}(Q') = Q$.  Then $\sigma(\tau(\sigma^{-1}))(Q') = Q'$, $\sigma \tau \sigma^{-1} \in D'$, so $\sigma D \sigma^{-1} \subseteq D'$.

Conversely assume $\tau' \in D'$, so $\tau'(Q') = Q'$ and $\tau'(\sigma(Q)) = Q'$. Thus $\sigma^{-1} \tau'(\sigma(Q)) = Q$, so $\sigma^{-1} \tau' \sigma \in D$ and $\tau' \in \sigma D \sigma^{-1}$.  Therefore $D' \subseteq \sigma D \sigma^{-1}$; we conclude $D = \sigma D \sigma^{-1}$.

{\bf $E' = \sigma E \sigma^{-1}$: } Let $\tau \in E$; then $\modequiv{\alpha}{\tau(\alpha)}{Q}$ for all $\alpha \in S$, so $\alpha - \tau(\alpha) \in Q$.  In particular $\sigma^{-1}(\alpha) - \tau(\sigma^{-1})(\alpha) \in Q$, and so $\alpha - \sigma(\tau(\sigma^{-1}(\alpha))) \in \sigma Q = Q'$.  Therefore $\sigma \tau \sigma^{-1} \in E'$ and $\sigma E \sigma^{-1} \subset E'$.

Conversely assume $\tau' \in E'$; then $\alpha - \tau'(\alpha) \in Q'$ for any $\alpha \in S$.  So $\sigma(\alpha) - \tau'(\sigma(\alpha)) \in Q'$ and $\alpha - \sigma^{-1}(\tau'(\sigma(\alpha))) \in \sigma^{-1}Q' = Q$.  Therefore $\sigma^{-1} E' \sigma \subset E$ and so $E' \subset \sigma E \sigma^{-1}$; we conclude $E = \sigma E' \sigma^{-1}$.

\item [9. (b)] Let $\modequiv{\psi(\alpha)}{\alpha^{||P||}}{Q}$; then $\psi(\alpha) - \alpha^{||P||} \in Q$ for any $\alpha \in S$.  In particular $\psi(\sigma(\alpha)) - \sigma(\alpha)^{||P||} \in Q$ and so $\sigma^{-1}\psi(\sigma(\alpha)) - \alpha^{||P||} \in \sigma^{-1}Q = Q'$.  Therefore $\psi' = \sigma^{-1}\psi \sigma$.

\item [12 (a)]  Let $\w = e^{2\pi i / m}$, let $G = \gal{\Q[\omega]}{\Q} \simeq \Z^{\times}_{m}$, $K$ be any subfield of $\Q[\omega]$ and $H$ the subgroup of $G$ fixing $K$.  Let $p \in \Z$ such that $p \nmid m$, and let $f$ denote the least integer such that $\overline{p^f} \in H$.  Show $f$ is the intertial degree $f(P | p)$ for any $P$ of $K$ lying over $p$.

Since $p$ is unramified, $\phi(P | p)$ exists and corresponds to some $b \in G$.  A permutation in $G$ corresponds to taking $\w \mapsto \w^b$.  However the Frobenius automorphism has the property $b(\omega) = \modequiv{\w^{b}}{\w^{p}}{P}$, so we have $\modequiv{b}{p}{n}$.

Let $H\overline{a_1}, \ldots, H\overline{a_n}$ be the cosets of $H$ (with $\overline{a_1}$ being the permutation that takes $\w \mapsto \w^{a_1}$).

Let the cyclic group $\{1, p, p^2, \ldots, p^{f-1} \}$ (with $p^{f} = 1$) act on the right cosets of $H$.  For any coset $Hx$, $p^{a}(Hx) = Hxp^{a}$; if $xp^{a} = x$ then $\w^{xp^{a}} = \w^{x}$ and so $\modequiv{xp^{a}}{x}{m}$.  Therefore $\modequiv{p^a}{1}{m}$; as $p \nmid m$, $a = 0$.  By the orbit stabilizer theorem, the size of an orbit of $Hx$ is the size of the whole group, i.e. $f$, by Theorem 33 the inertial degree of any prime $P$ of $K$ lying over $p$ is $f$.

\item [12. (b)] Let $\Q[\w + \w^{-1}]$ be a subfield of $Q$.  The subgroup of $G$ that fixes $\Q[\w + \w^{-1}]$ is the subgroup of order 2 consisting of the identity and complex conjugation $\tau$ that takes $\omega \mapsto \omega^{-1}$.  This $\tau$ is identified with the permutation $m - 1 \in \Z^{\times}_{m}$, so the subgroup $H$ consists of $\{1, m - 1 \} = \{1, -1 \}$.

Let $K = \Q[\w + \w^{-1}$.  By the tower law, $\phi(m) = [\Q[\w] : \Q] = [\Q[\w] : K][K : \Q] = 2 \cdot[K : \Q]$, so $[K : \Q] = \phi(m) / 2$.  For any odd prime such that $p \nmid m$, $p$ is unramified in $\Q[\w]$ and so also in $K$.  By (a), $p$ will split into $\phi(m) / 2f$ primes in $K$, where $f$ is the smallest integer such that $\modequiv{p^f}{\pm 1}{m}$.

\item [12. (c)] Letting $p$ be a prime not dividing $m$, and take $K$ to be any quadratic subfield $\Q[\sqrt{d}] \subset \Q[\w]$.  Let $H$ be the subgroup fixing $K$. Since $p \nmid m$, $p$ is unramified in $\Q[\w]$ and so also unramified in $\Q[\sqrt{d}]$.  Therefore either $p$ remains inert or splits into two primes in $\Q[\w]$.

Let $p$ be odd. We will show that $\bar{p} \in H$ iff $d$ is a square mod $p$; this is equivalent to $f = 1$ iff $d$ is a square mod $p$.  As $p$ is unramified, Theorem 25 gives that $f = 1$ iff $d$ is a square mod $p$.

Now let $p = 2$.  We will show that $\bar{p} \in H$ iff $\modequiv{d}{1}{8}$.  This is equivalent to $f = 1$ iff $\modequiv{d}{1}{8}$.  As $p$ is unramified, Theorem 25 gives the only possibility where $f = 1$ is $\modequiv{d}{1}{8}$.

\item [13.] Let $m \in \Z$.  Assume $m$ is not square and $m \neq -1$.  Let $K = \Q[\sqrt[4]{m}]$ and $L = \Q[\sqrt[4]{m}, i]$; then $L$ is a normal extension of $\Q$ such that $K \subset L$.  Denote the roots $\alpha, i\alpha, -\alpha, -i\alpha$ as 1, 2, 3, and 4.

\item [13. (a)] $[K : \Q] = 4$ and $[L : K] = 2$ so $L$ is a degree 8 extension of $\Q$ where $f(x) = x^4 - m$ splits.  Let $G = \text{Gal}(L / K)$; so $|G| = 8$.  Letting the roots of $f$ be $\alpha, i\alpha, -\alpha, -i\alpha$, $G \subset S_4$ and so $G \simeq D_8$ (the dihedral group on 4 objects) as this is the only subgroup of $S_4$ of order 8.  Therefore $G = \{ 1, \tau, \sigma, \tau \sigma, \sigma^2, \tau \sigma^2, \sigma^3, \tau \sigma^3 \}$.  In $D_8$, $\tau^2 = 1$ and $\sigma^4 = 1$, with $\tau \sigma = \sigma^{-1} \tau$.  Note $\tau$ corresponds to complex conjugation (switching $i\alpha$ with $-i\alpha$).

\item [13. (b)] Let $p$ be an odd prime not dividing $m$.  Prove $p$ is unramified in $L$.

Let $S = \ringofintegers{L}$ and consider $\disc{\alpha}$.  $\disc{\alpha} = \norm(f'(\alpha)) = \pm \norm(4\alpha^3) = \pm 4^8 \norm(\alpha)^3 = \pm 4^8 (-m)^3$; therefore $p \nmid \disc{\alpha}$.  Because $\disc{R} \mid \disc{\alpha}$, $p \nmid \disc{R}$ also, so $p$ is unramified in $L$.

\item [13. (c)] Let $Q$ be a prime lying over $p$.  Since $p$ is unramified in $L$, the Frobenius automorphism $\phi(Q|p)$ exists.  The subgroup $H$ of $G$ that fixes $K$ is the subgroup corresponding to complex conjugation: $\{1, \tau \}$.  The right cosets of $H$ are $H, H\sigma, H\sigma^2, H\sigma^3$.

Suppose $\phi(Q|p) = \tau$. Since $H\sigma\tau = H\tau \sigma^3 = H\sigma^3$, these two cosets are in the same partition.  $H\tau = H$ and $H\sigma^2 \tau = H\tau \sigma^2 \tau = H\sigma^2$.  Therefore we have three partitions of cosets: $\{H\}, \{H\sigma, H\sigma^3 \}, \{H\sigma^2\}$, and by Theorem 32, $Q$ splits into 3 primes in $K$.

\item [13. (d)] For each permutation of $G$, we follow a similar process to (c) to give how $Q$ splits.  The subgroup $H$ fixing $K$ remains the same as in (c).

The partitions are straightforward to calculate since right-multiplication of any $H\sigma^{n}$ by any permutation gives another coset of the form $H\sigma^{m}$.  This is straightforward for permutations $\sigma^a$, whereas for permutations $\sigma^a \tau$, \[H\sigma^n(\sigma^a \tau) = H\sigma^{n + a}\tau = H\tau\sigma^{-(n + a)} = H\sigma^{-(n + a)} \]

\begin{tabular}{|l|l|r|}
    \hline
    $\psi(Q|p)$ & Partitions & Number of Primes \\
    \hline
    1 & $\{H\}, \{H\sigma\}, \{H\sigma^2\}, \{H\sigma^3\}$ & 4 \\
    $\sigma$, $\sigma^3$ & $\{H, H\sigma, H\sigma^2, H\sigma^3\}$ & 1\\
    $\sigma^2$ & $\{H, H\sigma^2\}, \{H\sigma, H\sigma^3\}$ & 2\\
    $\tau$ & $\{H\}, \{H\sigma, H\sigma^3\}, \{H\sigma^2\}$ & 3 \\
    $\sigma\tau$ & $\{H, H\sigma^3\}, \{H\sigma, H\sigma^2\}$ & 2 \\
    $\sigma^2\tau$ & $\{H, H\sigma^2\}, \{H\sigma\}, \{H\sigma^3\}$ & 3 \\
    $\sigma^3\tau$ & $\{H, H\sigma\}, \{H\sigma^2, H\sigma^3\}$ & 2 \\
    \hline
\end{tabular}

% \item[14.] Let $\w = e^{2\pi i / m}$.  Take $m = p^k n$ where $p \nmid n$.  Then we can identify the Galois group of $\Q[\w]$ with $\gal{\Q[\omega]}{\Q} \simeq \Z^{\times}_{m} \simeq \Z^{\times}_{p^k} \times \Z^{\times}_n$.  Describe $D$ and $E$ corresponding to $p$ in terms of this direct product.

% By Theorem 26, the ramification degree of $p$ is $\phi(p^k)$ and the inertial degree of $p$ is its multiplicative order modulo $n$.  Therefore $E$ is a subgroup of $G$ of order $\phi(p^k)$ and $D$ is a subgroup of $G$ of order $\phi(p^k)f$.

% TODO

% Mirroring the proof of Theorem 26, $Q[\w]$ contains the subfields $\Q[\w^n]$ and $\Q[\w^{p^k}]$.  $p$ is ramified with degree $\phi(p^k)$ in $\Q[\w^n]$ and so

% In $\w^n$, any permutation that takes $\w^n \mapsto {\w^{n p^a}} = \w^{p^a} \w^{n}$

\end{enumerate}

\section*{Chapter 5}

\begin{enumerate}
\item[6.] Show that $\ringofintegers{\Q[\sqrt{m}]}$ is principal for $m = 2, 3, 5, 6, 7, 173, 293, 437$.  As each of these $m$ is positive, these are real quadratic fields and so the number of complex embeddings $s = 0$.  Therefore the bound given by Minowski's Theorem is that every ideal class contains an ideal with $||J|| < \frac{2!}{2^2} \cdot \sqrt{|\disc{R}|}$, where $\disc{R} = m$ if $\modequiv{m}{1}{4}$ and $\disc{R} = 4m$ otherwise.  Therefore \[ ||J|| < \begin{cases}\frac{\sqrt{m}}{2} &\modequiv{m}{1}{4}\\ \sqrt{m} &\modequiv{m}{2, 3}{4}\end{cases} \]

    \begin{itemize}
        \item[$m = 2$: ] $||J|| < \sqrt{2} \approx 1.4$ so every ideal class contains an ideal with norm 1.  Therefore every ideal must be principal.
        \item[$m = 3$: ] $||J|| < \sqrt{3} \approx 1.7$ so every ideal is principal.
        \item[$m = 5$: ] $||J|| < \sqrt{5} / 2 \approx 1.1$ so every ideal is principal.
        \item[$m = 6$: ] $||J|| < \sqrt{6} \approx 2.4$ so we must check that the prime ideal containing $2$ is principal.  By Theorem 25, 2 factors as $(2, \sqrt{6})^2$ in $\Q[\sqrt{6}]$.  $(2, \sqrt{6})$ is principal, generated by $(2 + \sqrt{6})$: $2 = -1 \cdot (2 + \sqrt{6})(2 - \sqrt{6})$ and $\sqrt{6} = 2 + \sqrt{6} - 2$.  Therefore every ideal is principal.
        \item[$m = 7$: ] $||J|| < \sqrt{7} \approx 2.6$ so we must check that the prime ideal containing $2$ is principal.  By Theorem 25, $2$ factors as $(2, 1 + \sqrt{7})^2$.  $(2, \sqrt{7})$ is generated by $3 + \sqrt{7}$; $2 = (3 + \sqrt{7})(3 - \sqrt{7})$ and $1 + \sqrt{7} = 3 + \sqrt{7} - 2$.  Therefore every ideal is principal.
        \item[$m = 173$: ] $\modequiv{173}{1}{4}$, so $||J|| < \sqrt{173} / 2 \approx 6.5$, so we must check that the prime ideals containing $2, 3, 5$ are all principal.  Since $\modequiv{173}{5}{8}$, $2$ remains prime in $\Q[\sqrt{m}]$ and so its ideal is principal.  173 is a prime number.  Since $\modequiv{173}{1}{4}$, by quadratic reciprocity, $\legendre{173}{p} = \legendre{p}{173}$, so $\legendre{3}{173} = \legendre{173}{3} = \legendre{2}{3} = -1$. Similarly $\legendre{5}{173} = \legendre{173}{5} = \legendre{3}{5} = -1$.  So both 3 and 5 remain inert in $\Q[\sqrt{m}]$ and so every ideal is principal.
        \item[$m = 293$: ] $\modequiv{293}{1}{4}$ so $||J|| < \sqrt{293} / 2 \approx 8.5$ so we must check the prime ideals containing $2, 3, 5, 7$ are all principal.  $\modequiv{293}{5}{8}$ so $2$ remains prime in $\Q[\sqrt{293}]$ and its ideal is principal.  293 is a prime number.  Calculation with Sage shows that $3, 5, 7$ each are not squares mod 293, so these prime ideals remain inert in $\Q[\sqrt{293}]$ and so are principal.  Therefore every ideal is principal.
        \item[$m = 437$: ] $\modequiv{437}{1}{4}$ so $||J|| < \sqrt{437} / 2 \approx 10.4$ so we must check that the prime ideals $2, 3, 5, 7$ are all principal.  $\modequiv{437}{5}{8}$ so $2$ remains prime in $\Q[\sqrt{437}]$.  $437 = 19 \cdot 23$ so none of $3, 5, 7$ ramify in $\sqrt{437}$.  Using Sage we can calculate the Jacobi symbol $\legendre{3, 5, 7}{437} = -1$.  As the Jacobi symbol is -1 each of these are nonresidues mod $437$ and so by Theorem 25 remain prime in $\Q[\sqrt{437}]$.  Therefore every ideal is principal.
    \end{itemize}

\end{enumerate}

\end{document}

